% ------------------------------------------------------------------------
% ------------------------------------------------------------------------
% ICMC: Modelo de Trabalho Acadêmico (tese de doutorado, dissertação de
% mestrado e trabalhos monográficos em geral) em conformidade com
% ABNT NBR 14724:2011: Informação e documentação - Trabalhos acadêmicos -
% Apresentação
% ------------------------------------------------------------------------
% ------------------------------------------------------------------------

% Opções:
%   Qualificação          = qualificacao
%   Curso                 = doutorado/mestrado
%   Situação do trabalho  = pre-defesa/pos-defesa (exceto para qualificação)
%   Versão para impressão = impressao
\documentclass[mestrado, pre-defesa, draft]{packages/icmc}

% ---------------------------------------------------------------------------
% Pacotes Opcionais
% ---------------------------------------------------------------------------
\usepackage{rotating}           % Usado para rotacionar o texto
\usepackage[all,knot,arc,import,poly]{xy}   % Pacote para desenhos gráficos
% Este pacote pode conflitar com outros pacotes gráficos como o ``pictex''
% Então é necessário usar apenas um dos pacotes conflitantes
\newcommand{\VerbL}{0.52\textwidth}
\newcommand{\LatL}{0.42\textwidth}
% ---------------------------------------------------------------------------


% ---
% Informações de dados para CAPA e FOLHA DE ROSTO
% ---
% Tanto na capa quanto nas folhas de rosto apenas a primeira letra da primeira palavra (ou nomes próprios) devem estar em letra maiúscula, todas as demais devem ser em letra minúscula.
\tituloPT{Geradores de homologia persistente e aplicações}
\tituloEN{Persistent homology generators and applications}
\autor[Ronchi, C. H. V.]{Carlos Henrique Venturi Ronchi}
\genero{M} % Gênero do autor (M = Masculino / F = Feminino)
\orientador[Orientador]{Prof. Dr.}{Marcio Fuzeto Gameiro}
\curso{MAT}
\data{16}{06}{2018} % Data do depósito
\idioma{PT} % Idioma principal do documento (PT = português / EN = inglês)
% ---

% include necessary packages
\usepackage{packages/carlos_package}

% ---
% RESUMOS
% ---

% Resumo em PORTUGUÊS
% conter no máximo 500 palavras
% conter no mínimo 1 e no máximo 5 palavras-chave
\textoresumo[brazil]{
    a.
    }{Modelo, Monografia de qualificação, Dissertação, Tese, Latex}


% resumo em INGLÊS
% conter no máximo 500 palavras
% conter no mínimo 1 e no máximo 5 palavras-chave
\textoresumo[english]{
    a.
    }{Template, Qualification monograph, Dissertation, Thesis, Latex}


% ----------------------------------------------------------
% ELEMENTOS PRÉ-TEXTUAIS
% ----------------------------------------------------------

% Inserir a ficha catalográfica
\incluifichacatalografica{tex/pre-textual/ficha-catalografica.pdf}

% DEDICATÓRIA / AGRADECIMENTO / EPÍGRAFE
%\textodedicatoria*{tex/pre-textual/dedicatoria}
%\textoagradecimentos*{tex/pre-textual/agradecimentos}
%\textoepigrafe*{tex/pre-textual/epigrafe}

% Inclui a lista de figuras
\incluilistadefiguras

% Inclui a lista de tabelas
\incluilistadetabelas

% Inclui a lista de quadros
%\incluilistadequadros

% Inclui a lista de algoritmos
\incluilistadealgoritmos

% Inclui a lista de códigos
\incluilistadecodigos

% Inclui a lista de siglas e abreviaturas
%\incluilistadesiglas

% Inclui a lista de símbolos
%\incluilistadesimbolos

% ----
% Início do documento
% ----
\begin{document}
% ----------------------------------------------------------
% ELEMENTOS TEXTUAIS
% ----------------------------------------------------------
\textual

\chapter{Introdução}
\label{chapter:introducao}
A topologia é a área da matemática que estuda formas e geometrias 
dos objetos através de funções contínuas. Um exemplo clássico 
é o da rosquinha e a caneca. Esses dois objetos tão distintos na vida
real se tornam iguais do ponto de vista topológico, já que existem 
tranformações que os torcem e entortam até o outro,
mas não cortam e colam. Para um estudo mais sistemático e algébrico,
a topologia algébrica foi desenvolvida. Com ela associamos
grupos ou espaços vetoriais a espaços topológicos e assim podemos extrair
informações como componentes conexas, buracos e cavidades, além
de outros buracos n-dimensionais. 

O estudo topológico dos objetos matemáticos com a álgebra é estudado 
desde o início do século XX com Poincaré~\cite{Poincare1895}. Mas 
apenas recentemente começou a se desenvolver algoritmos, devido ao
advento dos dados e computadores para cálculos eficientes. 
Essa nova área, chamada de análise topológica de dados, possui 
várias ferramentas, como a homologia persistente~\cite{edelsbrunner2010computational}
e o mapper~\cite{mapper}. 

A homologia persistente estuda os invariantes topológicos de dados 
através da topologia algébrica. Para cada conjunto de dados,
podemos construir um complexo simplicial, objeto combinatório da 
matemática que codifica informações do conjunto geometricamente. Com esse complexo,
podemos construir uma filtração e então obter informações
dos buracos n-dimensionais do conjunto de dados através dos
grupos de homologia, que geralmente são espaços vetoriais, e que por fim
são codificados no diagrama de persistência. Essa técnica
pode ser considerada uma ferramenta de redução de dimensão, assim como
para a extração de propriedades geométricas
de dados em baixa dimensão, como no plano ou espaço 3D.

Devido a natureza da ferramenta, nesta dissertação mostramos algumas
aplicações em biologia. O problema de enovelamento de proteína
pode ser resumido com a seguinte expressão:
\begin{center}
    Sequência de aminoacidos $\rightarrow$ Estrutura $\rightarrow$ Função.
\end{center}
A questão fundamental é como encontrar uma estrutura estável com uma específica
função a partir de uma sequência de aminoácidos específica.\cite{Dill2008}
Vários métodos foram desenvolvidos envolvendo propriedades físicas e estatísticas
das proteínas para prever a estrutura de uma molécula dada uma sequência de aminoácidos.
O mais conhecido é o software Rosetta, desenvolvido por um grupo da Universidade 
de Washington em 1997.\cite{Simons1997}

Este software tem tido muito sucesso na predição de proteínas pequenas, mas não é
perfeito. Várias vezes ele apresenta proteínas não estáveis, o que em desenvolvimento
de proteínas em larga escala pode acarretar em custos mais altos quando testes são realizados
no laboratório. Em \cite{Rocklin2017} apenas algumas proteínas são escolhidas
para a fase de testes em laboratório após seu modelamento no Rosetta. O método
de escolha é dado por um algoritmo de machine learning, que prevê a estabilidade
das moléculas baseado em 110 propriedades por proteína.

Como aplicação nós desenvolvemos um novo método para a escolha
de proteínas para a fase de testes que não apenas prevê a estabilidade em nível
do estado da arte, mas também nos dá novas informações. Para cada proteína, temos
alguns diagramas de persistência. Cada ponto desse diagrama de persistência
está relacionado com alguma cadeia na classe de homologia que pode ser 
visualizada na molécula. Temos então essa nova informação geométrica
que pode auxiliar no design de novas proteínas. 

Uma outra aplicação é o desenvolvimento de uma função de energia para o software 
Rosetta baseado em outras proteínas. Essa nova função de energia foi treinada
para prever a distância da molécula simulada em relação a original. Diversas proteínas
do PDB (Protein Data Bank) foram usadas no treinamento.

A dissertação está dividida nos seguintes capítulos. No Capítulo~\ref{chapter:hp101}
temos uma introdução computacional de homologia persistente, apresentando as principais
filtrações e como obter as informações topológicas através do algoritmo de redução. 
No Capítulo~\ref{chapter:mph} temos a fundamentação teórica de homologia persistente,
através de ideias inspiradas na teoria de medida e categorias. O Capítulo~\ref{chapter:miscel}
contém algoritmos decorrentes da homologia persistente para a utilização no aprendizado
de máquinas e também há uma introdução do mapper. Já no Capítulo~\ref{chapter:aplicacoes}
apresentamos duas aplicações voltadas ao desenvolvimento de proteínas, mostrando resultados
novos e similares aos do estado da arte. E o Capítulo~\ref{chapter:conclusao} fecha a dissertação
com algumas conclusões.  



\chapter{Homologia Persistente 101}
\label{chapter:hp101}
A topologia sempre foi vista como uma área de abstração da matemática, sem
espaço para aplicações. Ela é usada para o estudo de diversos espaços
em sua forma abstrata, auxiliando matemáticos em diversas demonstrações
de teoremas e dando uma base fundamental para grande parte da teoria matemática
usada no dia a dia~\cite{Poincare1895}.

Certas propriedades dos espaços topológicos são estudadas através da
topologia algébrica, dando algumas informações, como o número de componentes
conexas por caminhos de um espaço e buracos. A princípio esta é uma área altamente
abstrata da matemática,  nos últimos anos esta visão foi mudando,
com o desenvolvimento da Homologia Persistente e Análise Topológica de Dados.

Um conjunto de dados, geralmente um subconjunto finito de algum espaço métrico,
pode ser estudado através da homologia persistente e assim obtemos informações
topológicas do objeto em estudo.

O pipeline da análise topológica de dados pode ser divido nos seguintes passos:
\begin{itemize}
  \item A entrada do algoritmo pode ser um conjunto de pontos ou alguma matriz
  de distância/similaridade do conjunto de dados.
  \item A construção de um objeto combinatorial em cima do conjunto de dados ou
  da matriz de distância. Geralmente uma filtração ou um complexo simplicial.
  \item A partir da filtração ou do complexo simplicial é possível extrair informações
  topológicas e geométricas do conjunto de dados, por exemplo o número de
  componentes conexas, como um algoritmo de Clustering.
  \item Por fim a interpretação dos dados obtidos e possível pós processamento
  para a utilização em outros algoritmos, como os de classificação ou regressão.
\end{itemize}

Neste capítulo descrevemos de forma ingênua a homologia persistente, começando com
filtrações, passando pelos espaços vetoriais associados
aos complexos simpliciais e chegando ao algoritmo de homologia persistente.
Mostraremos também como interpretar os resultados obtidos.
A~\autoref{fig:pipeline_hp} mostra os passos para utilizar esta ferramenta em um conjunto de dados.

\begin{figure}
  \centering
  \includegraphics[width=0.7\textwidth]{images/pipeline_hp.png}
  \caption{Representação do pipeline para a utilização da homologia persistente
          com um conjunto de dados.}
  \label{fig:pipeline_hp}
\end{figure}


\section{Filtrações}
A filtração de um conjunto de dados é o primeiro passo na nossa sequência apresentada
na~\autoref{fig:pipeline_hp}.
Dado um conjunto de dados precisamos construir um objeto combinatorial de forma
que possa ser analisado do ponto de vista da topologia assim como computacionalmente.
A filtração é este objeto que captura as mudanças do conjunto dada uma escala.

Algumas definições se fazem necessárias para entendermos o que é a filtração
e qual o seu papel na análise topológica de dados. Começamos definindo um simplexo,
primeiro objeto combinatorial que é a base da filtração.

\begin{defi}
  Sejam $v_0, v_1, \dots, v_k \in \mathbb{R}^n$ linearmente afins, ou seja $\{v_1 - v_0, \dots, v_k - v_0\}$
  é um conjunto linearmente independente. O k-simplexo definido pelos pontos acima,
  chamados de vértices, é o conjunto

  \begin{equation*}
    \Set{\sum_{i=0}^k \lambda_i v_i \ | \ \sum_{i=0}^k \lambda_i = 1 \text{ e }
          \lambda_i \ge 0, \ \forall i}.
  \end{equation*}
\end{defi}

Note que para $k = 0$, temos um único vértice. Para $k=1$, temos uma reta, já
para $k=2$ temos um triângulo preenchido. E no caso $k=3$, um tetraedro. Os
simplexos podem ser vistos na~\autoref{fig:ksimpl}. Além disso, dizemos que a dimensão
do $k$-simplexo é $k$. A envoltoria convexa de qualquer subconjunto dos vértices
de um simplexo $S$ é chamado de face de $S$.

\begin{figure}
  \centering
  \includegraphics[width=0.7\textwidth]{images/ksimpl.png}
  \caption{Exemplos de $k$-simplexos para $k\in \Set{0,1,2,3}$.}
  \label{fig:ksimpl}
\end{figure}

Tendo definido os $k$-simplexos, podemos definir o complexo simplicial.
\begin{defi}
    Um complexo simplicial $K$ é uma coleção de simplexos satisfazendo as seguintes
    relações:
    \begin{itemize}
      \item Dado $\sigma \in K$, temos que para toda face $\tau \subset \sigma$
            vale $\tau \in K$.
      \item A interseção de dois simplexos é face de ambos os simplexos, em outras palavras,
      $\sigma, \tau \in K$ implica que $\sigma \cap \tau \subset \sigma$ e
      $\sigma \cap \tau \subset \tau$.
    \end{itemize}
\end{defi}
A segunda condição é necessária para evitar casos patológicos como mostrado na
\autoref{fig:simp_path}. Dizemos que a dimensão do complexo simplicial $K$ é a maior dimensão dentre os
simplexos em $K$. Podemos definir agora a filtração de um complexo simplicial.


\begin{figure}
  \centering
  \includegraphics[width=0.7\textwidth]{images/simp_path.png}
  \caption{Exemplo em que a interseção de dois simplexos não é um simplexo.}
  \label{fig:simp_path}
\end{figure}

\begin{defi}
  Seja $K$ um complexo simplicial. Definimos uma filtração de $K$ sendo uma
  sequência de subconjuntos $K_i \subset K$, com $i \in \{ 1, \dots, n \}$,
  de tal forma que $K_i$ é um complexo simplicial para todo $i$ e vale que
  \begin{equation*}
    K_1 \subset \dots \subset K_{n-1} \subset K_n = K.
  \end{equation*}
\end{defi}
Na~\autoref{fig:filtracao_exemplo} temos um exemplo de filtração para um complexo
simplicial.

\begin{figure}[hbt]
  \centering
  \includegraphics[width=0.7\textwidth]{images/filtracao_exemplo.png}
  \caption{Exemplo de filtração para um complexo simplicial $K$.}
  \label{fig:filtracao_exemplo}
\end{figure}


\subsection{Filtração de Čech}

\subsection{Filtração de Vietoris-Rips}

\subsection{Filtração \textit{Alpha Shape}}

\section{A matriz de bordo $\partial$}

\section{Redução da matriz}


\chapter{Módulos de Persistência}
\label{chapter:mph}
A homologia persistence teve seu ínicio em uma intersecção entre as ciências da computação 
e a matemática. Os primeiros artigos mostravam algoritmos sobre espaços topológicos simples,
como esferas \cite{Edelsbrunner2000}. No entanto, a teoria foi se desenvolvendo 
ao longo dos anos ao ponto em que as linguagens utilizadas para tratar da homologia persistente 
é a teoria de categorias conjuntamente com a teoria de representações \cite{Chazal2016}.

Neste capítulo tratamos do desenvolvimento da homologia persistente sob a luz dessas linguagens. 
Na primeira seção definimos o que são os módulos de persistência e suas relações com os diagramas de 
persistência. Na segunda seção descrevemos a medida retangular, usada para abstrair o conceito 
de diagrama de persistência e poder estudar o quão \textit{tame} ele o é. Apresentamos na terceira seção
alguns exemplos do comportamento dos módulos de persistência e exemplos. A quarta seção é fundamental,
pois mostramos como comparar dois módulos de persistência, através do \textit{interleaving}. E finalmente,
apresentamos o teoria de isometria e mostramos uma das implicações com a teoria desenvolvido neste capítulo. 

\section{Módulos de persistência e decomposições}
Nesta seção iremos definir os módulos de persistência, apresentar teoremas de decomposição dos módulos 
e introduzir a notação de quiver, que será utilizada para as próximas seções e demonstrações de 
outros resultados. 

Fixaremos aqui o corpo $\mathbf{k}$ para todos os espaços vetoriais apresentados neste texto. 

\begin{defi}
    Um módulo de persistência $\mathfrak{V}$ sobre os números reais $\mathbb{R}$ é uma família indexada 
    sobre $\mathbb{R}$ de espaços vetoriais 
    \begin{equation*}
        (V_t \mid t \in \mathbb{R}), 
    \end{equation*} 
    e uma família de aplicações lineares duplamente indexadas
    \begin{equation*}
        (v_t^s \colon V_s \to V_t \mid s \leq t) 
    \end{equation*}
    que satisfazem a seguinte relação de composição
    \begin{equation*}
        v_t^s \circ v_s^r = v_t^r,
    \end{equation*} 
    em que a função $v^r_r$ é considerada a função identidade. 
\end{defi}

O módulo de persistência pode ser visto como um funtor entre a categoria dos números reais com o morfismo
$s \to t$, em que $s \leq t$ e a categoria de espaços vetoriais. 

Vamos dar um exemplo de módulo de persistência que se encontra no contexto de análise topológica de dados. 
Seja $X$ um espaço vetorial e $f \colon X \to \mathbb{R}$ uma função, não necessariamente contínua e 
considere os conjuntos de nível
\begin{equation*}
    X^t = (X,f)^t = \Set{x \in X \mid f(x) \leq t}.
\end{equation*}

Temos uma sequência de conjuntos encaixados, $X^t$ com $t \in \mathbb{R}$, ou seja, existe uma função 
inclusão $\iota_t^s \colon X^s \hookrightarrow X^t$ que satisfaz trivialmente a lei de composição e 
existe uma função identidade. Chamamos esta sequência de conjuntos e funções de filtração de subníveis
de $(X,f)$, denotada por $\mathfrak{X}_{sub}$ ou $\mathfrak{X}^f_{sub}$.

Dada a sequência acima, podemos transforma-la em um módulo de persistência utilizando qualquer funtor
da categoria de espaços topológicos para a categoria de espaços vetoriais. Neste caso utilizamos 
o funtor de homologia $H = H_k(-, \mathbf{k})$ de dimensão $k$ com coeficientes em $\mathbf{k}$. Assim,
podemos definir o seguinte módulo de persistência $\mathfrak{V}$

\begin{equation*}
    V_t = H(X^t) \qquad v^s_t = H(\iota_t^s) \colon V_s \to V_t.
\end{equation*}
Podemos também escrever $\mathfrak{V} = H(\mathfrak{X}_{sub})$. 

Um exemplo na análise topológica de dados é quando $X$ é um complexo simmplicial finito e $X^t$ é um 
subcomplexo. Devido as propriedades dos complexos, existem finitos valores críticos onde há mudanças 
em $X$. Suponha que os valores sejam $a_1 < \dots < a_n$. Entao toda a informação do módulo de 
persistência é dada pela seguinte sequência de espaços vetoriais de dimensão finita

\begin{equation*}
    H(X^{a_1}) \to \dots \to H(X^{a_n}).
\end{equation*}

Neste caso, $H(\mathfrak{X}_{sub})$ admite uma descrição compacta, existe um algoritmo eficiente para 
o seu calculo e por último, a descrição é contínua com relação a $f$, ou seja, é estável sob uma 
métrica. 

A descrição mencionada acima é o diagrama de persistência ou barcode. A estrutura é dada por uma 
lista de intervalos da forma $[b,d) = [a_i, a_j)$ ou $[a_i, +\infty)$. Cada intervalo representa
um ciclo, uma propriedade, que nasce em $b$ e morre em $d$. 

Iremos mostrar aqui que é possível associar um diagram de persistência para módulos de 
persistência $\mathfrak{V}$ \textit{q-tame}. Um módulo de persistência é \textit{q-tame} 
se 
\begin{equation*}
    r_t^s = \text{rank}(v_t^s) < \infty \text{ para } s < t.
\end{equation*}
Intuitivamente falando, um módulo é \textit{q-tame} se para todo quadrante que pegamos com a origem
na diagonal, existem finitos pontos do diagram de persistência neste quadrante como pode ser visto
na \autoref{fig:quad_finito}.

\begin{figure}
    \centering
    \includegraphics[width=0.5\textwidth]{images/placeholder.png}
    \caption{Exemplo de um diagram de persistência de um módulo de 
            persistência \textit{q-tame} com um quadrante em destaque.}
    \label{fig:quad_finito}
\end{figure}

\subsection{Indíces e posets} 
No início desta seção definimos o módulo de persistência com o conjunto de indíces sendo os reais. No 
entanto, é possível definir utilizando quaisquer conjuntos parcialmente ordenados da mesma forma que 
com os reais. Seja $\mathbf{T}$ um poset, a coleção de espaços vetoriais e aplicações lineares que 
satisfazem as leis de composição e identidade é chamada de $\mathbf{T}$-módulo de persistência, ou 
módulo de persistência sobre $\mathbf{T}$. 

Além disso, podemos restringir o poset $\mathbf{T}$ para um subconjunto $\mathbf{S} \subset \mathbf{T}$
de forma a obter o $\mathbf{S}$-módulo de persistência, que são os espaços vetoriais e aplicações lineares
cujos indíces são elementos de $\mathbf{S}$. Esta é a restrição de $\mathfrak{V}$ em $\mathbf{S}$ e pode
ser denotada por $\mathfrak{V}_S$ ou $\left.\mathfrak{V}\right|_S$. 

\subsection{Categoria de módulos}
Com a definição de módulos de persistência sobre um poset $\mathbf{T}$ qualquer, podemos definir
homomorfismos entre módulos. Sejam $\mathfrak{U}, \mathfrak{V}$ $\mathbf{T}$-módulos de persistência.
Um homomorfismo $\Phi$ entre $\mathfrak{U}$ e $\mathfrak{V}$ é uma família de aplicações lineares 
$(\phi_t \colon U_t \to V_t \mid t \in \mathbf{T})$ tal que o seguinte diagrama comuta para todo
$s \leq t$. 

\begin{equation*}
    \begin{tikzcd}
        U_s \arrow{d}[swap]{\phi_s} \arrow{r}{u_t^s} & U_t \arrow{d}{\phi_t} \\
        V_s \arrow{r}{v^s_t}                     & V_t                    
    \end{tikzcd} 
\end{equation*}

A composição de dois homomorfismo $\Phi, \Psi$ é dada por cada indíce $t \in \mathbf{T}$, ou seja,
$\Phi \circ \Psi$ é a coleção de aplicações lineares $(\phi_t \circ \psi_t \colon U_t \to W_t \mid t \in \mathbf{T})$,
onde $\Phi$ é homomorfismo entre $\mathfrak{U}$ e $\mathfrak{V}$ e $\Psi$ entre $\mathfrak{V}$ e $\mathfrak{W}$. 
A identidade é definida de forma trivial. Portanto, temos a categoria dos módulos. Definamos os seguintes conjuntos
\begin{align*}
    \text{Hom}(\mathfrak{U}, \mathfrak{V}) &= \Set{\text{homomorfismos } \mathfrak{U} \to \mathfrak{V}}, \\
    \text{End}(\mathfrak{V}) &= \Set{\text{homomorfismos } \mathfrak{V} \to \mathfrak{V}}.
\end{align*}

\subsection{Módulos Intervalares}
A relação entre os diagramas de persistência e módulos de persistência são fundamentadas pelos módulos intervalares. 
Eles são a base da teoria de homologia persistente. 

Um intervalo em um conjunto totalmente ordenado $\mathbf{T}$ é um subconjunto $J \subset \mathbf{T}$ tal que se 
$r \in J$ e $t \in J$ tal que $r < s < t$, então $s \in J$. Portanto, para qualquer intervalo $J \subset \mathbf{T}$,
o módulo intervalar $\mathfrak{I} = \mathbf{k}^J$ é definido como o $\mathbf{T}$-módulo de persistência com a 
seguinte família de espaços vetoriais
\begin{equation*}
    I_t = \left\{
    \begin{split}
        & \mathbf{k} \text{ se } t \in J \\
        & 0 \text{ caso contrário,}
    \end{split}
    \right.
\end{equation*} 
e as aplicações lineares
\begin{equation*}
    i_t^s = \left\{
    \begin{split}
        & id \text{ se } s,t \in J \\
        & 0 \text{ caso contrário.}
    \end{split}
    \right.
\end{equation*}
Como mencionado anteriormente, os intervalos seriam as propriedades representadas no diagrama de persistência, 
ou seja, o módulo intervalar $\mathbf{k}^J$ representa uma propriedade que persiste por todo intervalo $J$.

Devida a sua importância, módulos intervalares com índices em subconjuntos de $\mathbb{R}$ possuem uma notação 
especial. Para distinguir os vários casos de intervalos, usamos uma supernotação: $+$ e $-$, a decoração dos pontos.
Para intervalos finitos adota-se o seguinte dicionário

\begin{align*}
    (p^-, q^-) &= [p,q) \\ 
    (p^-, q^+) &= [p,q] \\
    (p^+, q^-) &= (p,q) \\
    (p^+, q^+) &= (p,q]
\end{align*}
O dicionário acima vale para $p < q$. No caso em que $p=q$, representamos o intervalo por $(p^-, p^+) = [p,p]$. 
Para intervalos infinitos, usamos o símbolo $-\infty^+$ e $+\infty^-$ com definição similar à acima e com a adição
do seguinte intervalo
\begin{equation*}
    (-\infty^+, +\infty^-) = (-\infty, +\infty).
\end{equation*}
Quando queremos referenciar um ponto decorados mas não sabemos sua decoração, denotamos por $p^*$, podendo ser 
$p^-$ ou $p^+$. 

Podemos extender os reais para os reais decorados, um conjunto totalmente ordenado com as seguintes relações
\begin{equation*}
    p^- < p < p^+ < q^- < q < q^+,
\end{equation*}
para todo $p < q$. Definimos o semiplano diagonal superior em $\mathbb{R}^2$ como 
\begin{equation*}
    \mathcal{H} = \Set{(p,q) | p \leq q}
\end{equation*}
O semiplano diagonal superior $\bar{\mathcal{H}}$ é a união de $\mathcal{H}$ com os pontos no infinito. 

Portanto, um módulo intervalar pode ser representado de diversas formas, visualizados também na \autoref{fig:mod_int}
\begin{itemize}
    \item Como um intervalo na reta real;
    \item como uma função $\mathcal{H} \to \Set{0,1}$ definida por $(s,t) \mapsto \text{rank}(i^s_t)$; 
    \item como um ponto $(p,q) \in \mathcal{H}$ e um traço representando a respectiva decoração.
\end{itemize}
 
\begin{figure}[htpb!] 
    \centering
    \includegraphics[width=0.7\textwidth]{images/ex_interval_module.png}
    \caption{Representação por intervalo (esquerda), pela função rank (meio) e pelo ponto decorado (direita) do 
            módulo intervalar $\mathbf{k}[1,3) = \mathbf{k}(1^-, 3^-)$.}  
    \label{fig:mod_int}
\end{figure}
Os traços representando a decoração são dados por
\begin{center}
    $(p^-,q^-): $ \rotatebox[origin=c]{225}{$\multimap$}

    $(p^-,q^+): $ \rotatebox[origin=c]{135}{$\multimap$}

    $(p^+,q^-): $ \rotatebox[origin=c]{315}{$\multimap$}

    $(p^+,q^+): $ \rotatebox[origin=c]{45}{$\multimap$}
\end{center}

\subsection{Decomposição em módulos intervalares} 

\begin{defi}
    A \textbf{soma direta} $\mathfrak{W} = \mathfrak{U} \oplus \mathfrak{V}$ de dois módulos
    de persistência $\mathfrak{U}$ e $\mathfrak{V}$ é definida por
    \begin{equation*}
        W_t = U_t \oplus V_t, \quad w^s_t = u_t^s \oplus v^s_t
    \end{equation*}
\end{defi}
Esta definição generaliza-se para somar arbitrárias, tanto finitas como infinitas. Vamos agora definir
a indecomponibilidade de um módulo de persistência.

\begin{defi}
    Um módulo de persistência $\mathfrak{W}$ é indecomponível se dada uma decomposição $\mathfrak{U}
    \oplus \mathfrak{V}$, então $\mathfrak{U} = 0$ ou $\mathfrak{W}$ e $\mathfrak{V} = 0$ ou $
        \mathfrak{W}$.
\end{defi}

Podemos estudar os módulos de persistência através de sua decomposição por módulos intervalares. Dado
uma sequência de intervalos $(J_l \mid l \in L)$,
\begin{equation*}
    \mathfrak{V} \cong \bigoplus_{l \in L} \mathbf{k}^{J_l}.
\end{equation*}
Neste caso, podemos pensar que cada intervalo $J_l$ representa uma propriedade. Esta decomposição acaba
sendo muito importante por este motivo. Mas a questão que fica é: quais módulos são decomponíveis em 
intervalos? E porque decompõe-se em módulos intervalares? 

A resposta para a primeira pergunta é o Teorema \ref{teo:crawley}. Já para a segunda questão,
os módulos intervalares são indecomponíveis, como mostramos na Proposição \ref{prop:mod_int}.

\begin{propo}
    Seja $\mathfrak{I} = \mathbf{k}^J_T$ um módulo intervalar sobre $\mathbf{T} \subset \mathbb{R}$. 
    Então $\text{End}(\mathfrak{I}) = \mathbf{k}$. 
\end{propo}
\begin{proof}
    Vamos definir uma função $\Phi$ entre $\text{End}(\mathfrak{I})$ e $\mathbf{k}$ que será um isomorfismo
    de aneis.
    Seja $\Phi \colon \mathbf{k} \to \text{End}(\mathfrak{I})$ definida por 
    \begin{equation*}
        \alpha \mapsto \varphi^{\alpha}
    \end{equation*}
    onde $\varphi^\alpha$ é um endomorfismo de $\mathfrak{I}$ tal que $\varphi_t^{\alpha} \colon I_t 
    \to I_t$ e $\varphi^{\alpha}_t (x) = \alpha x$. É fácil ver que a aplicação é um homomorfismo de anéis.
    Vamos definir a inversa de $\Phi$. Para isso, note primeiro que qualquer endomorfismo de $\mathfrak{I}$
    age como multiplicação por escalar em qualquer $I_t$ não nulo. Precisamos mostrar que dados $s,t$, temos 
    que o escalar definido é o mesmo para ambos os casos:
    \begin{align*}
        \Psi^{-1} \colon & \mathfrak{I} \to \mathbf{k} \\
                         & \varphi \mapsto \alpha,
    \end{align*}
    Vamos mostrar que a aplicação está bem definida.
    
    Primeiro, pela observação acima, dados $s,t$ tais que $I_s, I_t \neq 0$, temos que vale o seguinte
    para $\varphi \in \mathfrak{I}$.
    \begin{align*}
        \varphi_s \colon & \mathbf{k} \to \mathbf{k} \\
                         &     x \mapsto \alpha x 
    \end{align*}
    e 
    \begin{align*}
        \varphi_t \colon & \mathbf{k} \to \mathbf{k} \\
                         &     x \mapsto \beta x 
    \end{align*}
    Precisamos mostrar que $\alpha = \beta$, demonstrando a proposição. Mas isso segue pelo 
    diagrama comutativo dos homomorfismos entre módulos de persistência, como podemos ver na 
    Eq. \eqref{eq:diag_prop_end}, assumindo que $s \leq t$. 
    \begin{equation}
        \label{eq:diag_prop_end}
        \begin{tikzcd}
            I_s \arrow{d}[swap]{\varphi_s} \arrow{r}{id} & I_t \arrow{d}{\varphi_t} \\
            I_s \arrow{r}{id}                     & I_t                    
        \end{tikzcd} 
    \end{equation}
    No caso acima temos a identidade entre $I_s$ e $I_t$, já que ambos são $\mathbf{k}$. Logo,
    segue que $\alpha=\beta$, provando a Proposição.
\end{proof}

\begin{propo}\label{prop:mod_int}
    Módulos intervalares são indecomponíveis. 
\end{propo}
\begin{proof}
    Suponha que exista uma decomposição $\mathfrak{I} = \mathfrak{U} \oplus \mathfrak{V}$. Considere agora
    as projeções sob $\mathfrak{U}$ e $\mathfrak{V}$. Ambas são homomorfismos idempotentes. Mas como 
    $\text{End}(\mathfrak{I})$ é isomorfo a $\mathbf{k}$ e os únicos idempotentes de $\mathbf{k}$ são 
    $0$ e $1$, segue que $\mathfrak{I}$ é indecomponível. 
\end{proof}

\begin{teo}{(Krull-Remak-Schmidt-Azumaya)}\label{teo:krull}
    Suponha que um módulo de persistência $\mathbf{T} \subset \mathbb{R}$ pode ser escrito como soma 
    direta de módulos intervalores de duas formas diferentes
    \begin{equation*}
        \mathfrak{V} \cong \bigoplus_{l \in L} \mathbf{k}^{J_l} \cong \bigoplus_{m \in M} \mathbf{k}^{K_m},
    \end{equation*}
    então existe uma bijeção $\sigma \colon L \to M$ tal que $J_l = K_{\sigma(l)}$ para todo $l \in L$. 
\end{teo}
\begin{proof}
    A demonstração segue do Teorema $1$~\cite{Azumaya1950} com a observação de que se $\mathbf{k}^J \cong 
    \mathbf{k}^L$, então $J=K$. Só é necessário verificar uma condição de localidade para aplicarmos o teorema:
    se $\psi, \phi \in \text{End}(\mathfrak{I})$ são não isomorfismos, então $\psi + \phi$ não é isomorfismo. 
    Mas pela proposição anterior, isso segue do fato que a unica aplicação que não é isomorfismo em 
    $\text{End}(\mathfrak{I})$ é a aplicação nula.
\end{proof}

\begin{teo}{(Gabriel, Auslander, Ringel-Tachikawa, Webb, Crawley-Boevey)}\label{teo:crawley}
    Seja $\mathfrak{V}$ um módulo de persistência sobre $\mathbf{T} \subset \mathbb{R}$. Então $\mathfrak{V}$
    pode ser decomposto como um soma direta de módulos intervalares sob as seguintes condições:
    \begin{itemize} 
        \item $\mathbf{T}$ é um conjunto finito;
        \item cada $V_t$ é um espaço vetorial de dimensão finita. 
    \end{itemize}
    Por outro lado, existe um módulo de persistência sob $\mathbb{Z}$ que não admite uma decomposição intervalar. 
\end{teo}  
\begin{proof}
    Detalhes podem ser vistos em \cite{Chazal2016}, página 22, \textbf{Teorema} 2.8.
\end{proof}

Se um módulo de persistência indexado sobre $\mathbb{R}$ pode ser decomposto
\begin{equation*}
    \mathfrak{V} \cong \bigoplus_{l \in L} \mathbf(p_l^*, q_l^*),
\end{equation*}
então o diagrama de persistência decorado é definido pelo multiconjunto
\begin{equation*}
    \text{Dgm} (\mathfrak{V}) = \text{Int}(\mathfrak{V}) = \Set{(p_l^*, q_l^*) | l \in L}
\end{equation*}
e o diagram de persistência não decorado é o multiconjunto
\begin{equation*}
    \text{dgm} (\mathfrak{V}) = \text{int}(\mathfrak{V}) = \Set{(p_l, q_l) | l \in L} - \Delta,
\end{equation*}
onde $\Delta$ é a diagonal no plano.

Note que ambos os diagramas definidos não dependem da escolha da decomposição, devido ao Teorema \ref{teo:krull}. 
Além disso, o diagrama dgm é o diagrama de pontos não decorados e sem a diagonal, sendo encontrado com 
frequência em exemplos práticos de análise de dados. Para a definição da distância bottleneck acaba sendo 
mais importante. 

\subsection{Cálculos com quivers}

Vamos agora definir uma notação para trabalhar com módulos de persistência sobre conjuntos de índices 
finitos. 
Um módulo de persistência $\mathfrak{V}$ sobre um conjunto finito de índices
\begin{equation*}
    \mathbf{T} : \qquad a_1 < \dots < a_n 
\end{equation*} 
da reta real pode ser visto como um diagrama de $n$ espaços vetoriais e $n-1$ aplicações lineares, como
mostrado abaixo
\begin{equation*}
    \mathfrak{V} : \quad V_{a_1} \longrightarrow \dots \longrightarrow V_{a_n}.
\end{equation*}
O diagrama acima é a representação do seguinte \textbf{quiver}:
\begin{equation*}
    \bullet \longrightarrow \bullet \longrightarrow \dots \longrightarrow \bullet
\end{equation*}

Vimos que podemos decompor alguns módulos de persistência em módulos intervalares. Para estes podemos representa-los 
com quivers da seguinte forma. 
\begin{ex}
    Seja $a < b < c$. Existem $6$ módulos intervalares diferentes sobre este intervalo.
\begin{equation*} 
    \begin{matrix}
        \mathbf{k}[a,a] = \qon{a} \qem \qoff{b} \qem \qoff{c} & \mathbf{k}[a,b] = \qon{a} \qem \qon{b} \qem \qoff{c}& 
                            \mathbf{k}[a,c] = \qon{a} \qem \qoff{b} \qem \qon{c}\\
        \mathbf{k}[b,b] = \qoff{a} \qem \qon{b} \qem \qoff{c}& \mathbf{k}[b,c] = \qoff{a} \qem \qon{b} \qem \qon{c}& \\
        \mathbf{k}[c,c] = \qoff{a} \qem \qoff{b} \qem \qon{c}& & & 
    \end{matrix}
\end{equation*}
\end{ex}

Os círculos $\bullet$ representam uma cópia do espaço vetorial $\mathbf{k}$ unidimensional. O círculo $\circ$ representa o 
espaço vetorial nulo. A aplicação linear entre dois $\bullet$ é a identidade e qualquer aplicação contendo $\circ$ é a
nula.

Esta notação pode ser usada para representar a multiplicidade dos módulos intervalares da decomposição
de um módulo de persistência sobre um conjunto de índices finito, essa quando existe. Seja $\mathfrak{V}$
um módulo de persistência sobre o conjunto $\mathbf{T} = \Set{a_1, \dots, a_n}$. Definimos a multiplicidade
de $[a_i, a_j] \subseteq \mathbf{T}$ em $\mathfrak{V}_{\mathbf{T}}$ como o número de cópias do módulo
$\mathbf{k}[a_i,a_j]$ na decomposição de $\mathfrak{V}_{\mathbf{T}}$. 

\begin{ex}
    Seja $\mathfrak{V}$ módulo de persistência sobre $\mathbf{T} = \Set{a,b,c}$. Escrevemos
    \begin{equation*}
        \innerproduct{[b,c] \mid \mathfrak{V}_{\mathbf{T}}} \text{ ou } \innerproduct{\qoff{a} \qem \qon{b} 
        \qem \qon{c} \mid \mathfrak{V}}
    \end{equation*}
    para representar a multiplicidade de $\qoff{a} \qem \qon{b} \qem \qon{c}$ no seguinte módulo de $3$ termos
    \begin{equation*}
        \mathfrak{V} : \quad V_{a_1} \longrightarrow \dots \longrightarrow V_{a_n}.
    \end{equation*}
\end{ex}

\begin{ex}
    Considere o módulo com dois espaços vetoriais e uma única aplicação linear $\mathfrak{V} : \quad 
    V_a \xrightarrow{\mu} V_b$. Então os invariantes de $\mu$ são
    \begin{align*}
        \text{rank}(\mu)       & = \innerproduct{\qon{a} \qem \qon{b} \mid \mathfrak{V}}, \\
        \text{nulidade}(\mu)   & = \innerproduct{\qon{a} \qem \qoff{b} \mid \mathfrak{V}}, \\
        \text{conulidade}(\mu) & = \innerproduct{\qoff{a} \qem \qon{b} \mid \mathfrak{V}}.
    \end{align*}
    Basta notar que para $V_a, V_b$ espaços de dimensão finita, existe uma decomposição das suas bases 
    \begin{equation*}
        e_1, \dots, e_r, f_1, \dots, f_n \text{   e   } g_1, \dots, g_r, h_1, \dots, h_m
    \end{equation*}
    tais que $\mu(e_i) = g_i$, $\mu(f_j) = 0$ para todo $i,j$. Assim, os espaços vetoriais unidimensionais
    gerados pelos elementos das bases geram uma decomposição do módulo $\mathfrak{V}$ nos seguintes intervalos
    \[\begin{array}{ccccc}
        (&\text{span}(e_i) & \rightarrow &\text{span}(g_i)&) \\
        (&\text{span}(f_j) & \rightarrow &0               &) \\
        (& 0               & \rightarrow &\text{span}(h_k)&)
    \end{array}\]
    que são isomorfos respectivamente à $\qon{a} \qem \qon{b}$, $\qon{a} \qem \qoff{b}$ e 
    $\qoff{a} \qem \qon{b}$. 
\end{ex}

\begin{propo}\label{teo:direct_sum}
    Suponha que podemos decompor um módulo de persistência $\mathfrak{V}$ como uma soma direta
    \begin{equation*}
        \mathfrak{V} = \bigoplus_{l\in L} \mathfrak{V}^l,
    \end{equation*}
    então 
    \begin{equation*}
        \langle [a_i, a_j] \mid \mathfrak{V}_{\mathbf{T}} \rangle = \sum_{l\in L} \langle [a_i, a_j] 
        \mid \mathfrak{V}_{\mathbf{T}}^l \rangle
    \end{equation*}
    para qualquer conjunto de índices $\mathbf{T} = \Set{a_1, \dots, a_n}$ e intervalos $[a_i, a_j] \subseteq \mathbf{T}$.
\end{propo}
\begin{proof}
    Segue do fato que a decomposição intervalar de $\mathfrak{V}_{\mathbf{T}}$ é a soma direta das decomposições intervalares 
    de cada $\mathfrak{V}^l_{\mathbf{T}}$ para todo $l \in L$. 
\end{proof}

\begin{propo}{(Princípio da restrição)}\label{teo:restr_princ}
    Sejam $\mathbf{S}, \mathbf{T}$ conjuntos de índices com $\mathbf{S} \subset \mathbf{T}$. Então
    \begin{equation*}
        \langle I \mid \mathfrak{V}_{\mathbf{S}} \rangle = 
        \sum_{J} \langle J \mid \mathfrak{V}_{J} \rangle,
    \end{equation*}
    onde a soma é sobre todos os intervalos $J \subseteq \mathbf{T}$ que se restringe a 
    $I$ sobre $\mathbf{S}$. 
\end{propo}
\begin{proof}   
    Tome uma decomposição intervalar arbitrária de $\mathfrak{V}_{\mathbf{T}}$. Então uma decomposição
    intervalar é induzida em $\mathfrak{V}_{\mathbf{S}}$. Note agora que para $I \subseteq \mathbf{S}$,
    temos diversos intervalos $J \subseteq \mathbf{T}$ tais que $J \cap \mathbf{S} = I$. Devido a 
    linearidade da soma direta, temos que os intervalos de $\mathfrak{V}_{\mathbf{S}}$ do tipo $I$ 
    são os intervalos de $\mathfrak{V}_{\mathbf{T}}$ do tipo $J$ acima. 
\end{proof}
\begin{ex}
    Seja $a < p < b < q < c$. Então temos os seguintes exemplos para os conjuntos de índices:
    \begin{itemize}
        \item $\mathbf{T}_1 = \Set{a,b,q,c}$, $S_1=\Set{a,b,c}$, $I_1=[b,c]$.
        \begin{equation*}
            \innerproduct{\qoff{a} \qem \qno \qem \qon{b} \qem \qno \qem \qon{c}} =
            \innerproduct{\qoff{a} \qem \qno \qem \qon{b} \qem \qon{q} \qem \qon{c}} 
        \end{equation*}
        \item $\mathbf{T}_2 = \Set{a,p,b,c}$, $S_2=\Set{a,b,c}$, $I_2=[b,c]$.
        \begin{equation*}
            \innerproduct{\qoff{a} \qem \qno \qem \qon{b} \qem \qno \qem \qon{c}} =
            \innerproduct{\qoff{a} \qem \qoff{p} \qem \qon{b} \qem \qno \qem \qon{c}}
            + \innerproduct{\qoff{a} \qem \qon{p} \qem \qon{b} \qem \qno \qem \qon{c}} 
        \end{equation*}
        \item $\mathbf{T}_3 = \Set{a,b,q,c}$, $S_2=\Set{a,b,c}$, $I_2=[c,c]$.
        \begin{equation*}
            \innerproduct{\qoff{a} \qem \qno \qem \qoff{b} \qem \qno \qem \qon{c}} =
            \innerproduct{\qoff{a} \qem \qno \qem \qoff{b} \qem \qoff{q} \qem \qon{c}}
            + \innerproduct{\qoff{a} \qem \qno \qem \qoff{b} \qem \qon{q} \qem \qon{c}} 
        \end{equation*}
         
    \end{itemize}
\end{ex}

\section{Medidas retangulares}
Na seção anterior discutimos módulos de persistência decomponíveis e seus diagramas de persistência,
Dgm e dgm. No entanto, nem sempre os módulos são decomponíveis, não sendo possível definir os 
diagramas de persistência. Para definir-los, podemos nos guiar pela seguinte ideia: se soubermos 
contar o número de pontos do Dgm pertence em cada retângulo do semiplano, então conhecemos Dgm.

Iremos nos inspirar na teoria da medida para construir uma função que nos dá um valor inteiro ou 
infinito e que podemos associar um diagrama de persistência com ela. A ideia é que para módulos 
bem comportados podemos avaliar esta função em retângulos e extrair um conjunto discreto de pontos,
que juntamente com sua multiplicidade gerará o diagrama de persistência. No caso que o módulo de 
persistência for decomponível, as definições são iguais. Caso contrário seguimos com a teoria 
normalmente. 

No resto deste capítulo iremos tratar apenas de medidas finitas no semiplano diagonal sem considerar
os pontos no infinito. Os argumentos usados podem ser facilmente estendidos para o caso de medidas
infinitas através de um processo de limite e para o semiplano diagonal com os pontos no infinito 
podemos usar o truque de colocar tudo dentro de um retângulo com a função $\arctan$.
 
\subsection{A medida de persistência}

\begin{defi} 
    Seja $\mathfrak{V}$ um módulo de persistência. Então a medida de persistência de $\mathfrak{V}$ é a função
    \begin{equation*}
        \mu_{\mathfrak{V}}(R) = \innerproduct{\qoff{a} \qem \qon{b} \qem \qon{c} \qem \qoff{d} \mid \mathfrak{V}}
    \end{equation*}
    definida no retângulo $R = [a,b]\times[c,d]$ no plano com $a < b \leq c < d$. 
\end{defi}

Veremos como a medida tem uma relação com módulos decomponíveis. Abaixo um resultado para módulos intervalares.

\begin{propo}\label{teo:int_meas}
    Seja $\mathfrak{V} = \mathbf{k}^J$, em que $J=(p^*, q^*)$ é um intervalo real. Seja $R=[a,b]\times[c,d]$
    tal que $a < b \leq c < d$. Então
    \begin{equation*}
        \mu_{\mathfrak{V}}(R) = \left\{
        \begin{split}
            1 & \text{ se } [b,c] \subseteq J \subseteq (a,d) \\
            0 & \text{ caso contrário.}
        \end{split}
        \right.
    \end{equation*}
\end{propo} 
\begin{proof}
    Como $\mathbf{k}^J$ restrito a $\Set{a,b,c,d}$ é só um intervalo ou o módulo nulo, temos que $\mu_{\mathfrak{V}}(R)
    \leq 1$, pois apenas teriamos a função identidade, cujo rank é um, ou a função nula, cujo rank é zero. 

    Como a medida tem valores em $\Set{0,1,\dots} \cup {\infty}$, vamos averiguar quando acontece $\mu_{\mathfrak{V}}(R)
    = 1$. Note que $\mu_{\mathfrak{V}}(R) = 1$ quando 
    \begin{equation*}    
        \mathbf{k}^J_{\{a,b,c,d\}} = \qoff{a} \qem \qon{b} \qem \qon{c} \qem \qoff{d}. 
    \end{equation*}
    E esta restrição vale se e somente se $b,c \in J$ e $a,d \nin J$.
\end{proof}

A Proposição \ref{teo:int_meas} pode ser representada visualmente. Considere o módulo intervalar como um ponto
decorado $(p^*, q^*)$ no semiplano diagonal superior. Então se o ponto estiver no interior do retângulo
$R$, ele será detectado independente da decoração. Mas se estiver na borda, apenas aqueles cuja decoração
apontem para dentro do retângulo serão detectados, como pode ser visto na \autoref{fig:dec_rect}
\begin{figure}[htpb!]
    \centering
    \includegraphics[width=0.5\textwidth]{images/placeholder.png}
    \caption{Pontos decorados que são detectados pela medida aplicada no retângulo R.}
    \label{fig:dec_rect}
\end{figure}

\begin{defi}
    Seja $R=[a,b]\times [c,d]$, em que $a < b \leq c < d$ e considere o ponto decorado $(p^*, q^*)$ com $p^* < q^*$.
    Definimos $(p^*, q^*) \in R$ se alguma das condições equivalentes é verdade:
    \begin{itemize}
        \item Se $p^* \in [a,b]$ e $q^* \in [c,d]$;
        \item Se $a < p^* < b$ e $c < q^* < d$ na ordem total definida anteriormente;
        \item Se $a^+ \leq p^* \leq b^-$ e $c^+ \leq q^* \leq d^-$;
        \item Se o intervalo satisfaz: $[b,c] \subseteq (p^*, q^*) \subseteq (a,d)$;
        \item O ponto com o traço $(p^*, q^*)$ está dentro do retângulo $R$.
    \end{itemize} 
\end{defi}

\begin{defi}
    Definimos com o \textbf{r-interior} do retângulo $R$ o conjunto
    \begin{equation*}
        R^\times = \Set{(p^*, q^*) | (p^*, q^*) \in R}.
    \end{equation*} 
    Também podemos definir o \textbf{interior} de $R$ como o conjunto
    \begin{equation*}
        R^\circ = (a,b) \times (c,d).
    \end{equation*}
\end{defi}

A expressão $|_R$ indica a restrição de um multiconjunto de pontos decorados no retângulo $R$.

\begin{cor}
    Suponha que $\mathfrak{V}$ seja um módulo de persistência decomponível sobre $R$.
    \begin{equation*}
        \mathfrak{V} = \bigoplus_{l \in L} k(p^*_l, q^*_l).
    \end{equation*}
    Então 
    \begin{equation*}
        \mu_{\mathfrak{R}}(R) = \text{card}(\text{Dgm}(\mathfrak{V}\left.)\right|_R)
    \end{equation*}
    para todo retângulo $R = [a,b] \times [c,d]$ com $a < b \leq c < d$. 
\end{cor}
\begin{proof}
   A demonstração segue direto das Proposições \ref{teo:int_meas} e \ref{teo:direct_sum}. 
\end{proof}

A função $\mu$ é chamada de medida pois é aditiva em relação a divisão dos retângulos. 
Vamos provar este fato agora.

\begin{propo}\label{teo:split_measure}
    $\mu_{\mathfrak{V}}$ é aditiva sobre divisão vertical e horizontal dos retângulos:
    \begin{align*}
        \mu_{\mathfrak{V}}([a,b]\times[c,d]) = \mu_{\mathfrak{V}}([a,p]\times[c,d]) 
        + \mu_{\frakv}([p,b] \times [c,d]) \\
        \mu_{\mathfrak{V}}([a,b]\times[c,d]) = \mu_{\mathfrak{V}}([a,b]\times[c,q]) 
        + \mu_{\frakv}([a,b] \times [q,d])
    \end{align*}
    para todo $a < p < b \leq c < q < d$. Esta propriedade pode ser visualizada na Figura 
    \ref{fig:split_measure}.
    \begin{figure}[htpb!]
        \centering
        \includegraphics[width=0.5\textwidth]{images/placeholder.png}
        \caption{Representação gráfica da Proposição \ref{teo:split_measure}}
        \label{fig:split_measure}
    \end{figure}
\end{propo}
\begin{proof}
    A demonstração segue direto da Proposição~\ref{teo:restr_princ}: para a aditividade na divisão
    horizontal temos que
    \begin{align*}
        \mu_{\frakv}([a,b] \times [c,d]) & = \innerproduct{\qoff{a} \qem \qno \qem \qon{b} \qem \qon{c} \qem \qoff{d}} \\
        &=\innerproduct{\qoff{a} \qem \qon{p} \qem \qon{b} \qem \qon{c} \qem \qoff{d}}
          + \innerproduct{\qoff{a} \qem \qoff{p} \qem \qon{b} \qem \qon{c} \qem \qoff{d}} \\
        &=\innerproduct{\qoff{a} \qem \qon{p} \qem \qno \qem \qon{c} \qem \qoff{d}}
          + \innerproduct{\qno \qem \qoff{p} \qem \qon{b} \qem \qon{c} \qem \qoff{d}} \\
        &=\mu_{\frakv}([a,p] \times [c,d]) + \mu_{\frakv}([p,b] \times [c,d]).
    \end{align*}
    De forma análoga vale o resultado para a divisão vertical.  
\end{proof}

\subsection{r-medidas abstratas}
Até agora trabalhamos com módulos de persistência e uma medida associada. Porém, podemos
trabalhar de forma mais abstrata, sem mencionar os módulos. 

\begin{defi}
    Seja $\mathcal{D} \subseteq \mathbb{R}^2$. Defina
    \begin{equation*}
        \text{Rect}(\mathcal{D}) = \Set{[a,b]\times[c,d] \subset \mathcal{D} | a < b \text{ e } c < d}.  
    \end{equation*} 
    A \textbf{r-medida} ou \textbf{medida retangular} em $\mathcal{D}$ é uma função
    \begin{equation*}
        \mu \colon \text{Rect}(\mathcal{D}) \to \Set{0,1,\dots} \cup \Set{\infty},
    \end{equation*}
    que também é aditiva na divisão horizontal e vertical dos retângulos. 
\end{defi}

\begin{propo}
    Seja $\mu$ uma \textbf{r-medida} em $\mathcal \subseteq \mathbb{R}^2$. Então 
    \begin{itemize}
        \item Se $R \in \text{Rect}(\mathcal{D})$ pode ser escrito como a união de retângulos com interior
        disjuntos, $R = R_1 \cup \dots \cup R_n$, então $\mu(R) = \sum \mu(R_i)$;
        \item Se $R \subseteq S$, então $\mu(R) \leq \mu(S)$.
    \end{itemize}    
\end{propo}
\begin{proof}
    (\textit{Finitamente aditiva}) Seja $R = [a,b] \times [c,d]$. Por indução e pela propriedade de divisão
    vertical, temos que a aditividade finita vale para decomposições da forma 
    \begin{equation*}
        R = \bigcup_i R_i,
    \end{equation*}
    em que cada $R_i = [a_i, a_{i+1}] \times [c,d]$, com $a = a_1 < \dots < a_m = b$. De forma análoga,
    vale para divisões horizontais e portanto a aditividade vale para decomposições da forma
    \begin{equation*}
        R = [a,b] \times [c,d] = \bigcup_{i,j} R_{ij},
    \end{equation*}
    onde $R_{ij} = [a_i, a_{i+1}] \times [c_j, c_{j+1}]$ com $a = a_1 < \dots < a_m = b$ e $c = c_1 < 
    \dots < c_n = d$. Para uma decomposição arbitrária $R = R_1 \cup \dots \cup R_k$ o resultado segue
    considerando uma decomposição em que cada $R_i$ é decomposto em intervalos da forma acima.  

    (\textit{Monotonicidade}) Decomponha $S$ em uma coleção de retângulos $R$ e $R_1, \dots, R_k$ que 
    possuem interiores disjuntos. Portanto, da propriedade de aditividade e que $\mu \geq 0$
    \begin{align*}
        \mu(S) & = \mu(R) + \mu(R_1) + \dots + \mu(R_k) \\
               & \geq \mu(R).
    \end{align*}
\end{proof}

\begin{propo}{(Subaditividade)} 
    Seja $\mu$ uma \textbf{r-medida} em $\mathcal{D} \subseteq \mathbb{R}^2$. Se um retângulo $R \in 
    \text{Rect}(\mathcal{D})$ está contido numa união finita de retângulos $R_i \in \text{Rect}(
    \mathcal{D})$
    \begin{equation*}
        R \subseteq R_1 \cup \dots \cup R_k,
    \end{equation*}
    então
    \begin{equation*}
        \mu(R) \leq \mu(R_1) + \dots + \mu(R_k).
    \end{equation*}
\end{propo}
\begin{proof}
    Seja $a_1 < \dots < a_m$ sequência de todos os valores do eixo $x$ de cada vértice dos retângulos. 
    Considere também $c_1 < \dots < c_n$ sequência de todos os valores do eixo $y$ de cada vértice dos retângulos.
    Então cada retângulo $R_k$ pode ser decomposto como união dos seguintes subretângulos
    \begin{equation*}
        [a_i, a_{i+1}] \times [c_j, c_{j+1}],
    \end{equation*}
    que possuem interiores disjuntos por construção e sua medida é a soma das medidas de cada subretângulo. 
    Como cada subretângulo de $R$ pertence a um ou mais retângulos $R_i$, segue da aditividade o resultado.
    A Figura~\ref{fig:dem_subadd} mostra uma demonstração visual desta proposição. 
    \begin{figure}[htpb!]
        \centering
        \includegraphics[width=0.7\textwidth]{images/placeholder.png}
        \caption{Demonstração da proposição através da figura.}
        \label{fig:dem_subadd}
    \end{figure}
\end{proof}

\subsection{Equivalência de medidas e diagramas}
As $r$-medidas de persistência permitem o estudo dos diagramas de persistência de maneira mais analítica,
facilitando o desenvolvimento da teoria. Nesta seção iremos demonstrar a equivalência entre medidas 
abstratas e multiconjuntos localmente finitos. Para isso, assumiremos que a medida é finita, como dito
anteriormente. 

O \textbf{$r$-interior} de uma região $\mathcal{D} \subseteq \mathbb{R}^2$ é o conjunto definido abaixo
\begin{equation*}
    \mathcal{D}^\times = \Set{(p^*, q^*) | \exists R \in \text{Rect}(\mathcal{D}) \text{ tal que }
    (p^*, q^*) \in R}.
\end{equation*}
A definição acima pode ser vista com o seguinte significado: o conjunto dos ponto decorados pode ser 
determinado por algum retâgulo em $\mathcal{D}$. 

O \textbf{interior} de $\mathcal{D}$ é dado por 
\begin{equation*}
    \mathcal{D}^\circ = \Set{(p,q) | \exists R \in \text{Rect}(\mathcal{D}) \text{ tal que } 
    (p,q) \in R^\circ},
\end{equation*}
onde para um retângulo $R = [a,b]\times[c,d]$, $R^\circ = (a,b)\times(c,d)$. 

\begin{teo}{(O teorema da equivalência)}\label{teo:equiv_meas}
    Seja $\mathcal{D} \subseteq \mathbb{R}^2$. Então existe uma correspondência bijetiva
    entre
    \begin{enumerate}
        \item $r$-medidas $\mu$ finitas em $\mathcal{D}$. Finito neste caso significa que 
        $\mu(R) < \infty$ para todo $R \in \text{Rect}(\mathcal{D})$.
        \item Multiconjuntos $A$ em $\mathcal{D}$ localmente finitos. Localmente finito significa
        $\text{card}(\left.A\right|_R) < \infty$ para todo $R \in \text{Rect}(\mathcal{D})$.
    \end{enumerate}
    A medida $\mu$ correspondente ao multiconjunto $A$ é relacionada pela fórmula
    \begin{equation}\label{eq:med_dgm}
        \mu(R) = \text{card}(\left.A\right|_R)
    \end{equation}
    para todo $R \in \text{Rect}(\mathcal{D})$, ou equivalentemente
    \begin{equation}\label{eq:med_mult}
        \mu(R) = \sum_{(p^*,q^*)\in R} m(p^*, q^*),
    \end{equation}
    em que
    \begin{equation*}
        m \colon \mathcal{D}^{\times} \to \Set{0,1,2,\dots}
    \end{equation*}
    é a função multiplicidade de $A$. 
\end{teo}
\begin{proof}
$(2) \to (1):$ para este passo, basta provar que a medida definida na Eq. \eqref{eq:med_dgm} é uma $r$-medida.
De fato, é finita pois $A$ é localmente finito. Para verificar a aditivade, suponha que para um retângulo
$R$ qualquer, ele se divida horizontalmente ou verticalmente em $R_1$ e $R_2$. Note então que $(p^*,q^*)$
pertence a exatamente $R_1$ ou $R_2$. Portanto, 
\begin{equation*}
    \mu(R) = \text{card}(\left.A\right|_R) = \text{card}(\left.A\right|_{R_1}) + \text{card}(\left.A\right|_{R_2})
     = \mu(R_1) + \mu(R_2),
\end{equation*}
provando a primeira implicação. 

\noindent$(1) \to (2):$ Dada uma $r$-medida, iremos (1) construir o multiconjunto $A$ em $\mathcal{D}^\times$,
(2) mostrar que $\mu$ e $A$ estão relacionadas pela Eq. \eqref{eq:med_dgm} e (3) mostrar que $A$ é único. Na 
prática, iremos construir a função de multiplicidade $m$ e definir a Eq. \eqref{eq:med_mult} ao invés de $A$ diretamente.

\textbf{Passo 1.} (Fórmula da multiplicidade) Seja $\mu$ uma $r$-medida finita em $\mathcal{D}$. Define
\begin{equation}
    m(p^*, q^*) = \min \Set{\mu(R) | R \in \text{Rect}(\mathcal{D}), (p^*, q^*) \in R}
\end{equation}  
para $(p^*, q^*) \in \mathcal{D}^\times$. Observe que o mínimo é atingido, já que tomamos $(p^*, q^*) \in \mathcal{D}^\times$
e $\mu$ é uma medida que assume valores inteiros. Utilizaremos uma definição alternativa, ao invés de minimizar sob 
todos os retângulos, tomamos o limite de uma sequência de retângulos decrescentes.  
\begin{lema}\label{teo:lem_med}
    Sejam $(\xi_i)$ e $(\eta_i)$ duas sequências não crescentes de números reais positivos
    que tendem a zero quando $i \to \infty$. Então
    \begin{equation*}
        m(p^+, q^+) = \lim_{i\to\infty} \mu([p,p+\xi_i] \times [q, q+\eta_i]),
    \end{equation*}
    e similarmente
    \begin{align*}
        & m(p^+, q^-) = \lim_{i\to\infty} \mu([p,p+\xi_i] \times [q-\eta_i, q]), \\
        & m(p^-, q^+) = \lim_{i\to\infty} \mu([p-\xi_i,p] \times [q, q+\eta_i]), \\
        & m(p^-, q^-) = \lim_{i\to\infty} \mu([p-\xi_i,p] \times [q-\eta_i,q]).
    \end{align*}
\end{lema}
\begin{proof}
    Note primeiro que a sequêncina de retângulos $R_i = [p, p+\xi_i] \times [q, q+\eta_i]$ é 
    cofinal no conjunto de retângulos $R$ contendo $(p^+, q^+)$, ou seja, para todo $R$ deste 
    tipo, $R_i \subseteq R$ para $i$ suficientemente grande.
    
    Pela monotonicidade de $\mu$ e como a sequência de inteiros não negativos $\mu(R_i)$ é não 
    crescente, ela estabiliza no seu limite em algum momento. Portanto
    \begin{equation*}
        m(p^+, q^+) \leq \min_i\mu(R_i) = \lim_{i\to\infty} \mu(R_i) \leq \mu(R)
    \end{equation*}
    para todo $R$ contendo $(p^+, q^+)$. Tomando o mínimo de ambos os lados da inequalidade acima 
    sob todos os $R$, o lado direito se torna $m(p^+, q^+)$, logo
    \begin{equation*}
        m(p^+, q^+) = \lim_{i\to\infty}\mu(R_i).
    \end{equation*}
    Os outros três casos são similares. 
\end{proof} 

\textbf{Passo 2.} Uma vez com a função multiplicidade definida, mostraremos que ela está de acordo com a 
Eq. \eqref{eq:med_mult}. Podemos definir uma medida baseada na função multiplicidade
\begin{equation*}
    \nu(R) = \sum_{(p^*, q^*)\in R} m(p^*, q^*),
\end{equation*}
nos resta mostrar então que $\mu = \nu$. Vamos utilizar indução sobre $k = \mu(R)$. 

\noindent \textbf{Caso base.} Se $\mu(R) = 0$, então para todo $(p^*, q^*) \in R$ temos 
\begin{equation*}
    0 \leq m(p^*, q^*) \leq \mu(R) = 0.
\end{equation*}
Portanto, $\nu(R) = 0$. 

\noindent \textbf{Passo indutivo.} Suponha qe $\mu(R) = \nu(R)$ para todo retângulo $R$ com 
$\mu(R) < k$. Seja agora um retângulo $R_0$ tal que $\mu(R_0) = k$. Vamos mostrar que $\nu(R_0)
=0$. A ideia para este passo é em construir uma sequência decrescente de retângulos fechados 
de forma que haverá apenas um ponto na interseção destes retângulos (Teorema de Cantor). 

Divida o retângulo em quatro quadrantes iguais, $S_1, \dots, S_4$. Pela aditividade finita
\begin{align*}
    \mu(R_0) & = \mu(S_1) + \dots + \mu(S_4) \\
    \nu(R_0) & = \nu(S_1) + \dots + \nu(S_4).
\end{align*}
Se todo quadrante satisfaz $\mu(S_i) < k$, segue então que $\mu(R_0) = \nu(R_0)$, concluindo
a demonstração. Caso contrário, um dos quadrantes tem valor $k$, enquanto o resto será $0$. 
Seja $R_1$ o quadrado especial, então $\mu(R_1) = k$. Resta mostrar que $\nu(R_1) = k$. 

Repetindo o argumento acima, divida o retângulo $R_i$ em quatro quadrantes iguais. Temos dois
casos: todos os quadrantes satisfazem a hipótese indutiva $\mu<k$ e teriamos terminado a situação.
Ou existe um quadrante $R_{i+1}$ com $\mu(R_{i+1})=k$ e temos que mostrar que $\nu(R_{i+1})$.

No pior caso, temos uma sequência decrescente de retângulos fechados
\begin{equation*}
    R_0 \supset R_1 \supset R_2 \supset \dots
\end{equation*}
com cada quadrante sendo do mesmo tipo anterior, $\mu(R_i) = k$. Pelo teorema de cantor e o fato
que o diâmetro dos conjuntos tende a $0$, temos que a interseção dos intervalos fechados possuem
apenas um ponto $(r,s)$.

Vamos mostrar agora que $\nu(R_0) = k$ avaliando a soma explicitamente sob todos os pontos decorados
em $R_0$.

Note primeiro que pontos decorados em $R_0$ que saem da sequência de retângulos em algum momento não
contribuem para $\nu(R_0)$, já que se $(p^*, q^*) \in R_0$ e $(p^*, q^*) \in R_{i-1} - R_i$ para 
algum $i$, então o ponto pertence a algum dos outros três quadrantes, sendo assim $\mu = 0$. 
Portanto, pela fórmula de multiplicidade, $m(p^*, q^*) = 0$. 

Assim, os únicos pontos que contribuem para $\nu(R_0)$ são as variações decoradas de $(r,s)$, já 
que é o único ponto não decorado na interseção. Agora a avaliação de $\nu(R_0)$ depende de como 
este ponto decorado se encontra na intereseção. Existem 3 possíveis casos: as 4, 2, 1 decorações
estão contidas nos retângulos, como podemos ver na Figura \ref{fig:proof_rect}

\begin{figure}[htpb!]
    \centering
    \includegraphics[width=0.7\textwidth]{images/placeholder.png}
    \caption{Possíveis casos dos pontos decorados $(r^*, s^*)$ na interseção $\cap_i R_i$.}
    \label{fig:proof_rect}
\end{figure}

Vamos mostrar apenas para o caso em que as 4 decorações estão em todos os retângulos $R_i$. Os outros
dois casos são análogos. 

Suponha que todas as decorações $(r^*, s^*) \in R_i$ para todo $i$. Agora divida cada retângulo
$R_i$ em quatro partes, de forma que cada parte contenha apenas uma decoração: $R_i^{++}, R_i^{+-},
R_i^{-+}, R_i^{--}$. A divisão ocorre de forma que o ponto $(r,s)$ fique num vértice dividido pelos 
quatro retângulos. Pelo Lema \ref{teo:lem_med},
\begin{align*} 
    & m(r^+, s^+) = \lim_{i \to \infty} \mu(R_i^{++}), \quad m(r^+, s^-) = \lim_{i \to \infty} \mu(R_i^{+-}), \\
    & m(r^-, s^+) = \lim_{i \to \infty} \mu(R_i^{-+}), \quad m(r^-, s^-) = \lim_{i \to \infty} \mu(R_i^{--}).
\end{align*}
Além disso, cada uma dessas sequências decrescentes de inteiros estabilizam no seu limite. Portanto, para valores
de $i$ suficientemente grandes
\begin{align*}
    \nu(R_0) & = m(r^+, s^+) + m(r^+, s^-) + m(r^-, s^+) + m(r^-, s^-) \\
             & = \mu(R_i^{++}) +  \mu(R_i^{+-}) + \mu(R_i^{-+}) + \mu(R_i^{--}) = \mu(R_i) = k. 
\end{align*}

\textbf{Passo 3.} (Unicidade) Suponha que $m'(p^*, q^*)$ é um outra função multiplicidade em $\mathcal{D}^\times$ 
cuja $r$-medida associada 
\begin{equation*}
    \nu'(R) = \sum_{(p^*, q^*) \in R} m'(p^*, q^*)
\end{equation*}
satisfaz $\mu = \nu'$. Vamos mostrar que $m=m'$. 

Seja $(p^*, q^*) \in \mathcal{D}^\times$ e $R$ um retângulo que contém o ponto $(p^*, q^*)$ em um
dos seus vértices. Como $\nu(R) = \nu'(R) = \mu(R) < \infty$, existem apenas finitos pontos decorados
em $R$. Além disso, podemos diminuir $R$ de forma que $(p^*, q^*)$ seja o único ponto decorado em $R$
com multiplicidade positiva em ambas as medidas. Portanto,
\begin{equation*}
    m(p^*, q^*) = \nu(R) = \mu(R) = \nu'(R) = m'(p^*, q^*).
\end{equation*} 
Como $(p^*, q^*)$ era um ponto qualquer, segue o resultado. 
\end{proof}

\begin{defi} 
    Seja $\mu$ uma medida finita em $\mathcal{D} \subseteq \mathbb{R}^2$. Então
    \begin{itemize}
        \item o \textbf{diagrama decorado} de $\mu$ é o único multiconjunto localmente finito
        $\text{Dgm}(\mu)$ em $\mathcal{D}^{\times}$ tal que 
        \begin{equation*}
            \mu(R) = \text{card}(\text{Dgm}(\mu\left.)\right|_R)
        \end{equation*}
        para todo retângulo $R \in \text{Rect}(\mathcal{D})$;
        \item o \textbf{diagrama não decorado} de $\mu$ é o multiconjunto localmente finito em 
        $\mathcal{D}^{\circ}$
        \begin{equation*}
            \text{dgm}(\mu) = \Set{(p,q) | (p^*, q^*) \in \text{Dgm}(\mu)} \cap \mathcal{D}^\times 
        \end{equation*}
        obtido esquecendo a decoração dos pontos e restringindo ao interior. 
    \end{itemize}
\end{defi}

\section{Comportamento de módulos e exemplos}

Quando extende-se as medidas e diagramas de persistência para o semiplano diagonal superior com os 
pontos no infinito e para medidas que assumem valor infinito também, os módulos de persistência não são
sempre bem comportados, necessitando diferenciar entre as diversas situações.
Abaixo estão três diferentes noções e em seguida apresentaremos apenas mais uma, já que estamos trabalhando
apenas com medidas finitas. 

\begin{itemize}
    \item Um módulo de persistência é do \textbf{tipo finito} se é uma soma direta finita de módulos intervalares;
    \item Um módulo de persistência é \textbf{localmente finito} se é uma soma direta de módulos intervalares
    e satisfaz a seguinte propriedade: qualquer subconjunto de $\mathbb{R}$ intersecta um número finito de 
    módulos intervalares;
    \item Um módulo de persistência $\frakv$ é \textbf{pontualmente de dimensão finita (pfd)} se cada espaço
    vetorial $V_t$ tiver dimensão finita. 
\end{itemize}

Como estamos trabalhando medidas finitas apenas, vamos definir o módulo \textit{q-tame}. Seja 
$\frakv$ um módulo de persistência. Dizemos que $\mathfrak{V}$ é \textit{q-tame} se 
$\mu_{\frakv}(Q) < \infty$ para todo quadrante $Q$ que não toca a diagonal. Em outras palavras
\begin{equation*}
    \innerproduct{\qon{b} \qem \qon{c} \mid \frakv} < \infty
\end{equation*}
para todo $b < c$. O diagrama de persistência $\text{Dgm}(\mu_\frakv)$ é definido sobre o conjunto
\begin{equation*}
    \Set{(p^*, q^*) | -\infty \leq p < q \leq + \infty}.
\end{equation*}

Abaixo mostramos alguns exemplos que encontra-se na teoria para módulos de persistência \textit{q-tame}.

\begin{teo}\label{teo:tameum}
Seja $X$ um poliedro compacto e $f \colon X \to \mathbb{R}$ uma função contínua. Então a homologia
persistente $H(\mathfrak{X}_{\text{sub}})$ da filtração de subnível de $(X,f)$ é \textit{q-tame}. 
\end{teo}
\begin{proof}
Precisamos mostrar que 
\begin{equation*}
    H(X^b) \to H(X^c) 
\end{equation*}
tem rank finito para qualquer $b < c$. Considere uma triangulação de $X$ e a subdivida de forma que 
nenhum simplexo intersecte $f^{-1}(b)$ e $f^{-1}(c)$ ao mesmo tempo. Defina agora $Y$ como a união
de todos os simplexos que intersectam $X^b$. Portanto, 
\begin{equation*}
    X^b \subseteq Y \subseteq X^c
\end{equation*} 
e aplicando o funtor de homologia
\begin{equation*}
    H(X^b) \longrightarrow H(Y) \longrightarrow H(X^c).
\end{equation*}
Como $Y$ é um poliedro compacto, $H(Y)$ tem dimensão finita, portanto a aplicação $H(X^b) \to H(X^c)$ 
tem rank finito. 
\end{proof}

\begin{cor}\label{teo:tamedois}
Seja $X$ um poliedro localmente compacto e $f \colon X \to \mathbb{R}$ uma função propriamente contínua
que seja limitada por baixo. Então $H(\mathfrak{X}_{\text{sub}})$ é \textit{q-tame}.
\end{cor}
\begin{proof}
Novamente, precisamos mostrar que $H(X^b) \to H(X^c)$ tem rank finito. Mas neste caso iremos aplicar
o Teorema~\ref{teo:tameum}, para tanto precisamos de um subpoliedro compacto de $X$ que contenha $X^c$. 
Seja uma triangulação local finita de $X$ e considere os simplexos fechados que intersectam $X^c$. 
Como $X^c$ é compacto, já que $f$ é uma função própria, no sentido que a pre-imagem de compacto é compacto,
temos que existe um número finito de conjuntos fechados que intersecta $X^c$. Considere agora a união
desses conjuntos fechados como o subpoliedro desejado. 
\end{proof}

\begin{cor}\label{teo:tametres} 
Seja $A$ um subconjunto não vazio compacto de $X = \mathbb{R}^n$ e $f \colon X \to \mathbb{R}$, $f(x) =
\min_{a\in A} \norm{x-a}$, para qualquer norma $\norm{.}$. Segue então do Corolário \ref{teo:tamedois}
que $H(\mathfrak{X}_{\text{sub}})$ é \textit{q-tame}.
\end{cor}

\section{\textit{Intercalação}} 

A intercalação é um modo de comparar dois módulos de persistência. Dizemos que os módulos
$\mathfrak{U}$ e $\frakv$ são isomorfos se existem homomorfismos
\begin{equation*}
    \Psi \in \text{Hom}(\mathfrak{U}, \frakv), \quad \Phi \in \text{Hom}(\frakv, \fraku), 
\end{equation*}
tais que 
\begin{equation*}
    \Psi \Phi = 1_\fraku, \quad \Phi \Psi = 1_\frakv.
\end{equation*}

No entanto, para trabalhar com módulos de persistência, a noção de isomorfismo é muito forte. 
Para isso, podemos enfraquece-la definindo a $\delta$-intercalação entre dois módulos. 
Nesta subseção iremos definir o interlaçamento e demonstrar o lema de interpolação,
como se define um \textit{caminho contínuo} entre dois módulos de persistência. 

\subsection{Homomorfismos e módulos de persistência}

Primeiro vamos considerar homomorfismo que mudam o índice de persistência dos módulos. 
Sejam $\fraku, \frakv$ módulos de persistência sobre $\mathbb{R}$ e $\delta \in \mathbb{R}$
qualquer. Então um homomorfismo de grau $\delta$ é a coleção $\Phi$ de aplicações lineares
\begin{equation*}
    \phi_t \colon U_t \to V_{t+\delta}
\end{equation*}
para todo $t \in \mathbb{R}$ tal que o diagrama 
\begin{equation*}
    \begin{tikzcd}[row sep=huge, column sep = huge]
        & U_s \rar{u_t^s} \dar[swap]{\phi_s} & U_t \dar{\phi_t} \\
        & V_{s+\delta} \rar{v^{s+\delta}_{t+\delta}} & V_{t+\delta}
    \end{tikzcd}
\end{equation*}
comuta para todo $s \leq t$. Escrevemos 
\begin{align*}
    \text{Hom}^\delta(\mathfrak{U}, \mathfrak{V}) &= \Set{\text{homomorfismos } \mathfrak{U} \to \mathfrak{V}}
    \text{ de grau } \delta, \\
    \text{End}(\mathfrak{V}) &= \Set{\text{homomorfismos } \mathfrak{V} \to \mathfrak{V}\text{ de grau } \delta}.
\end{align*}
A composição é definida de forma natural, nos dando a aplicação
\begin{equation*}
    \text{Hom}^{\delta_2}(\frakv, \mathfrak{W}) \times \text{Hom}^{\delta_1}(\fraku, \frakv) 
    \to \text{Hom}^{\delta_1 + \delta_2}(\fraku, \mathfrak{W}).
\end{equation*}
Para $\delta \geq 0$, a aplicação de grau $\delta$ mais importante é a aplicação
\textit{shift}
\begin{equation*}
    1^\delta_\frakv \in\text{End}^\delta(\frakv),
\end{equation*}
que é a coleção de aplicações lineares $(v_{t+\delta}^t)$ da estrutura de $\frakv$. Se $\Phi$ 
é um homomorfismo $\fraku \to \frakv$ de grau quaqluer, então por definição 
\begin{equation*}
    \Phi 1_\fraku^\delta = 1_\fraku^\delta \Phi,
\end{equation*}
para todo $\delta \geq 0$. 

\begin{propo}\label{teo:prop_inter}
Sejam $x,y$ números reais. Os módulos de persistência $\mathfrak{U}$ 
e $\frakv$ são $\abs{y-x}$-intercalados se, e somente se, existe 
um módulo de persistência $\mathfrak{W}$ sobre $\Delta_x \cup
\Delta_y$ tal que $\left.\mathfrak{W}\right|_{\Delta_x} = 
\mathfrak{U}$ e  $\left.\mathfrak{W}\right|_{\Delta_y} = 
\mathfrak{V}$. O conjunto $\Delta_x \cup \Delta_y$ é um subposet
de $\mathbb{R}^2$.  
\end{propo}
\begin{proof}

\end{proof}

\subsection{O lema de interpolação}

Vamos anunciar o lema da interpolação, que será utilizado na 
demonstração do teorema da estabilidade dos módulos de persistência.

\begin{lema}{(Lema de interpolação)}\label{teo:lema_interp}
Suponha que $\mathfrak{U}$ e $\frakv$ são módulos de persistência
$\delta$-interlaçados. Então existe uma família de módulos de 
persistência $(\mathfrak{U}_x \mid x \in [0,\delta])$ tais que 
$U_0$ e $U_\delta$ são iguais a $\mathfrak{U}$ e $\frakv$ 
respectivamente e $\mathfrak{U}_x$, $\mathfrak{U}_y$ são 
$\abs{y-x}$-interlaçados para todo $x,y \in [0,\delta]$. Além disso,
se $\mathfrak{U}$ e $\mathfrak{V}$ são \textit{q-tames}, então 
$(\mathfrak{U}_x)$ são \textit{q-tames}.
\end{lema}

O Teorema~\ref{teo:lema_interpdois} é uma versão mais forte do 
Lema~\ref{teo:lema_interp}. 
\begin{teo}{(Lema de interpolação, versão 2)}\label{teo:lema_interpdois}
Todo módulo de persistência sobre $\Delta_0 \cup 
\Delta_\delta, \mathfrak{W}$, se extende a um módulo de persistência 
$\bar{\mathfrak{W}}$ sobre a faixa diagonal
\begin{equation*}
    \Delta_{[0, \delta]} = \Set{(p,q) | 0 \leq q - p \leq 2\delta}
    \subset \mathbb{R}^2.
\end{equation*} 
Se $\left.\mathfrak{W}\right|_{\Delta_0}, \left.\mathfrak{W}
\right|_{\Delta_\delta}$ são \textit{q-tames}, então a extensão
pode ser escolhida de forma que cada $\left.\bar{\mathfrak{W}}
\right|_{\Delta_x}$ é \textit{q-tame}. 
\end{teo}
\begin{proof}
    Sejam $\Delta_0 \cup \Delta_\delta$ e $\Delta_{[0,\delta]}$ duas 
    categorias no mesmo sentido definido no início deste capítulo para
    o conjunto dos reais $\mathbb{R}$ ($s \to t \text{ sss } s \leq t$).
    Então os módulos de persistência sobre esses posets podem ser vistos
    como funtores para a categoria dos espaços vetoriais. O teorema
    de extensão de Kan~\cite{MacLane1978} afirma que existe uma extensão
    $\bar{\mathfrak{W}}$ 
    \begin{equation*}
        \begin{tikzcd}
            {\Delta_{[x_0,x_1]}} \arrow[dotted]{rd}{\bar{\mathfrak{W}}}     &             \\
            \Delta_{x_0} \cup \Delta_{x_1} \arrow{u} \arrow{r}{\mathfrak{W}}  & \text{Vect}
        \end{tikzcd}
    \end{equation*}
    para qualquer funtor $\mathfrak{W}$, completando
    a demonstração. 
    
\end{proof}

Se $\mathfrak{U}$ e $\frakv$ são $\delta$-interlaçados, então 
pela Proposição~\ref{teo:prop_inter} existe um módulo de 
persistência $\mathfrak{W}$ sobre $\Delta_0 \cup \Delta_\delta$ tal
que $\left.\mathfrak{W}\right|_{\Delta_0} = \mathfrak{U}$ e 
$\left.\mathfrak{W}\right|_{\Delta_\delta} = \frakv$. Pelo 
Teorema~\ref{teo:lema_interpdois}, este módulo se extende a 
$\bar{\mathfrak{W}}$ sobre a faixa $\Delta_{[0,\delta]}$.
Defina então $\mathfrak{U}_x = \left.\bar{\mathfrak{W}}\right|_{
\Delta_x}$, logo $\mathfrak{U}_x$ e $\mathfrak{U}_y$ são 
$\abs{y-x}$-interlaçados para todo $x,y\in [0, \delta]$. 

\section{O teorema de isometria}


\chapter{Geradores ótimos e outros conceitos}
\label{chapter:miscel}
\section{Geradores ótimos}

\section{Vetorização do diagrama de persistência}

\section{Mapper}


\chapter{A estabilidade da proteína}
\label{chapter:aplicacoes}
O problema de enovalemento da proteína é a questão fundamental de como a sua sequência de aminoácidos
no plano se transforma em uma estrutura atômica tridimensional. Esta questão é essencial, pois
um melhor entendimento dessa situação pode levar ao desenvolvimento de novos remédios e também
uma melhora no combate de doenças. No entanto, continua sendo um grande desafio obter
uma estrutura estável da proteína a partir da sequência de aminoácidos.
Recentemente, alguns estudos \cite{Rocklin2017} desenvolveram novas proteínas usando o softaware
\textit{Rosetta}, que modela estruturas macromolecures. Existem alguns problemas no
desenvolvimento usando tal software, como proteínas modeladas que não são tão
estáveis sobre o processo de proteólise, que é a quebra da proteína em pedaços
menores. Pode-se contornar este problema através do uso de outras ferramentas avançadas, como
as encontradas em aprendizado de máquinas e análise topológica de dados.

Neste capítulo estudamos a estabilidade de proteínas sobre um score proposto
em \cite{Rocklin2017} utilizando imagens de persistência \cite{Adams2017}, descritas
no Capítulo \ref{chapter:miscel}, e vários algoritmos de aprendizado de máquinas
implementados em \cite{scikit-learn}. Mais detalhes são dados na Seção \ref{sec:stabprot}

Por outro lado podemos estudar a perfomance de algoritmos para modelagem computacional e
análise estrutural de proteínas, como \textit{Rosetta} e \textit{Amber}.
Em \cite{Rubenstein2018}, ambos os softwares sõa comparados em relação a proteína-energia.
Para uma proteína específica, eles geraram milhares de moléculas similares e calcularem
a raíz do erro quadrático médio em relação a proteína original.
Após isso, eles analisaram as móleculas de falso mínimo. Dada uma lista de proteínas
simuladas, elas são ranqueadas de acordo com suas energias normalizadas. Uma molécula
simulada é uma falsa mínima se está no top $10$ das moléculas no ranking a seu
RMSD é maior do que $5$. Eles observaram que o software \textit{Rosetta} gerou mais
proteínas de falso mínimo do que o \textit{Amber}. Na Seção \ref{sec:predrmsd}
analisamos a estrutura das proteínas dadas pelo \textit{Rosetta} utilizando ciclos ótimos
\cite{Escolar2015}, imagens de persistência \cite{Adams2017}, VAE's \cite{kingma2013} e
diversos algoritmos de machine learning utilizando o sklearn \cite{scikit-learn}. Nós
tentamos prever e apresentar uma nova função para ajudar o \textit{Rosetta} no seu
passo de otimização quando simulando novas moléculas.

\section{Estudando a estabilidade - Proteínas I}\label{sec:stabprot}

O desenvolvimento de novas moléculas através de softwares acelerou o estudo de 
novas estruturas atômicas e suas propriedades. No entanto, fica cada vez mais díficil
verificar experimentalmente se uma proteína modelada computacionalmente é estável ou não,
pois por várias vezes existem milhares ou dezenas de milhares de moléculas, dificultando
a experimentação no laboratório devido ao grande número e o seu custo relacionado.

Em \cite{Rocklin2017} eles apresentam um novo método de desenvolvimento de proteínas
com o auxílio de algoritmos de aprendizado de máquinas. Para cada round de desenvolvimento
de proteínas, eles selecionam apenas algumas para testar a estabilidade no laboratório e 
estudar o que precisa ser modificado na estrutura da molécula para o próximo round de 
experimentos computacionais. Nesta seção apresentamos um método que utiliza 
análise topológica de dados para selecionar as moléculas com resultados similares. Porém, 
utilizando homologia persistente obtemos outra informações geométricas da proteína, que pode
auxiliar no desenvolvimento e aprimoramento em cada round de experimentos. Por exemplo, as informações 
dos buracos e cavidades, que podem ser obtidas com a homologia persistente, nos dizem o quão hidrofóbica 
ou hidrofílica uma proteína é. \cite{Jamadagni2011} 

\subsection{A estabilidade da proteína}
Dada uma proteína modelada, podemos medir a sua estabilidade da seguinte maneira. Primeiro,
deposita-se várias cópias da proteína sobre uma célula de levedura, para várias céulas. Então cada 
célula é geneticamente fundida à uma tag de expressão que pode ser tabelada com fluorescência. 
Assim, as células são encubadas com diversas concentrações diferentes de enzimas proteolíticas. 
A quantidade de enzimas utilizadas para quebrar metade das proteínas depositadas na levedura
são então gravadas e vão dar o score de estabilidade da proteína. Se um valor baseado 
nessa soma das concentrações for maior que 1, dizemos que a proteína é estável, caso contrário
dizemos que ela é instável. Podemos ver na Figura~\ref{fig:yeast_cell} o processo de quebra
das cópias da proteína. Mais detalhes podem ser vistos em \cite{Rocklin2017}.

\begin{figure}
    \centering
    \includegraphics[width=.8\textwidth]{images/yeast_cell.png}
    \caption{Processo de quebra de uma cópia da proteína sobre a superfície de uma 
            célula de levedura.}
    \label{fig:yeast_cell}
    \fautor
\end{figure}

\subsection{Prevendo a estabilidade}
O conjunto de proteínas utilizado contém 12927 moléculas para o treinamento e 
3232 proteínas para o teste. No conjunto de treinamento existem 2210 proteínas 
estáveis (com score maior do que 1) enquanto que no conjunto de treinamento 
existem 305 moléculas estáveis. 

Cada proteína tem 110 propriedades associadas, ou seja, valores numéricos,
que vão desde área de superfície não acessível (relacionada à hidrofobicidade
da proteína) a potenciais coulombianos.  

Os autores de \cite{Rocklin2017} desenvolveram um algoritmo para prever a estabilidade
da proteína e selecionar as melhores moléculas para testes em laboratório. Eles usaram 
o algoritmo de árvore de decisões \textit{Random forest} para treinar mais de 12000
proteínas e prever seu score de estabilidade. Foi treinado um regressor com o erro quadrático
médio (MSE em inglês) para a minimização. Os resultados são dados em relação à raíz do erro quadrático
médio (RMSE em inglês), erro porcentual (RMSE dividido pela diferença entre o maior e menor score
de estabilidade) e podem ser vistos na Tabela~\ref{tab:rockl_result}

\begin{table}[!htpb]
    \centering
    \caption{Resultados do algoritmo treinado pelos autores de \cite{Rocklin2017}.}
    \label{tab:rockl_result}
    \begin{tabular}{@{}ccc@{}}
    \toprule
    Modelo        & RMSE  & Erro Percentual (\%) \\ \midrule
    Random Forest & 0,419 & 11,381               \\ \bottomrule
    \end{tabular}
\end{table} 

A seguir apresentamos dois métodos utilizando homologia persistente para o problema de predição
do score de estabilidade. O primeiro foi usando as propriedades obtidas utilizando
homologia persistente e imagens de persistente além das propriedades das proteínas já geradas. 
Já o segundo foi apenas utilizando as imagens de persistência das proteínas. Nas próximas
seções descrevemos a metodologia utilizada, parâmetros e resultados.

\subsection{Metodologia}
Para cada proteína construímos sete subconjuntos, um para cada conjunto em $\mathcal{A} = \Set{\{C\}, \{O\}, \{N\}, 
\{C, O\}, \{C, N\}, \{O,N\}, \{C,O,N\}}$. Para cada subconjunto calculamos os diagramas de persistência
de dimensão 1 e 2 utilizando a filtração de Alpha. Os pesos utilizados para a filtração de Alpha foram os raios
de Van der Waals para cada átomo. 

Para vetorizar os diagramas de persistência utilizamos imagem de persistência com os seguintes parâmetros:
\begin{itemize}
    \item Tamanho da imagem: $5 \times 4$;
    \item Variância: $0,1,\quad 0,3,\quad 0,5,\quad 0,7$. 
\end{itemize}

Então, concatenamos as imagens de persistência de forma a obter um vetor de tamanho $280$. Para o
primeiro método concatenamos as 110 propriedades de cada proteína no final do vetor, totalizando 
o seu tamanho em $390$. Uma vez com os vetores podemos treinar os algoritmos de aprendizado de máquinas
para prever o score de estabilidade. Abaixo temos a liste de algoritmos utilizados:
\begin{itemize}
    \item Regressão linear;
    \item Regressão linear com regularização; 
    \item Árvore de decisão;
    \item GBoost. 
\end{itemize}
E por fim após o treinamento obtemos os scores de estabilidade dos conjuntos de teste. 
A Figura~\ref{fig:proteinpipeline} mostra o pipeline da metodologia utilizada. 

\begin{figure}[!htpb] 
    \centering
    \includegraphics[width=0.99\textwidth]{images/proteinpipeline.pdf}
    \caption{Pipeline da metodologia utilizada para a predição do score de estabilidade.} 
    \label{fig:proteinpipeline}
    \fautor 
\end{figure}

\subsection{Resultados e análises} 
\subsubsection{Primeiro método}
O primeiro método consiste em utilizar as propriedades da proteínas e as imagens de 
persistência para prever o score de estabilidade. Temos 4 tabelas para cada uma das 
variâncias. 

\begin{table}[htpb!]
    \centering
    \caption{Resultados para variância igual a $0,1$.}
    \label{tab:var01}
    \begin{tabular}{@{}ccccc@{}}
    \toprule
    Modelo                 & MSE    & RMSE   & Erro Percentual (\%) & $R^2$  \\ \midrule
    GBoost                 & 0,1831 & 0,4278 & 11,61                & 0,5529 \\
    Regressão Linear       & 0,2025 & 0,4500 & 12,21                & 0,5054 \\
    Regressão lin. c/ Reg. & 0,2084 & 0,4565 & 12,39                & 0,4910 \\
    Árvore de decisão I    & 0,1771 & 0,4208 & 11,42                & 0,5674 \\
    Árvore de decisão II   & 0,1780 & 0,4219 & 11,45                & 0,5653 \\ \bottomrule
    \end{tabular}
\end{table}
\begin{table}[htpb!]
    \centering
    \caption{Resultados para variância igual a $0,3$.}
    \label{tab:var03}
    \begin{tabular}{@{}ccccc@{}}
    \toprule
    Modelo                 & MSE    & RMSE   & Erro Percentual (\%) & $R^2$  \\ \midrule
    GBoost                 & 0,1832 & 0,4281 & 11,62                & 0.5525 \\
    Regressão Linear       & 23,6637 & 4,8645 & 132,01                & -56.7950 \\
    Regressão lin. c/ Reg. & 0,2075 & 0,4555 & 12,36                & 0,4932 \\
    Árvore de decisão I    & 0,1772 & 0,4209 & 11,42                & 0,5672 \\
    Árvore de decisão II   & 0,1785 & 0,4225 & 11,46                & 0,5641 \\ \bottomrule
    \end{tabular}
\end{table}
\begin{table}[htpb!]
    \centering
    \caption{Resultados para variância igual a $0,5$.}
    \label{tab:var05}
    \begin{tabular}{@{}ccccc@{}}
    \toprule
    Modelo                 & MSE    & RMSE   & Erro Percentual (\%) & $R^2$  \\ \midrule
    GBoost                 & 0,1829 & 0,4277 & 11,61 & 0,5533 \\
    Regressão Linear       & 0,2032 & 0,4508 & 12,23 & 0,5036 \\
    Regressão lin. c/ Reg. & 0,2068 & 0,4548 & 12,34 & 0,4949 \\
    Árvore de decisão I    & 0,1781 & 0,4220 & 11,45 & 0,5650 \\
    Árvore de decisão II   & 0,1783 & 0,4222 & 11,46 & 0,5646 \\
    \bottomrule
    \end{tabular}
\end{table}
\begin{table}[htpb!]
    \centering
    \caption{Resultados para variância igual a $0,7$.}
    \label{tab:var07}
    \begin{tabular}{@{}ccccc@{}}
    \toprule
    Modelo                 & MSE    & RMSE   & Erro Percentual (\%) & $R^2$  \\ \midrule
    GBoost                 & 0,1828 & 0,4275 & 11,60 & 0,5536 \\
    Regressão Linear       & 0,2009 & 0,4482 & 12,16 & 0,5093 \\
    Regressão lin. c/ Reg. & 0,2069 & 0,4549 & 12,34 & 0,4947 \\
    Árvore de decisão I    & 0,1779 & 0,4218 & 11,45 & 0,5655 \\
    Árvore de decisão II   & 0,1790 & 0,4230 & 11.48 & 0,5629 \\
    \bottomrule
    \end{tabular}
\end{table}

Esperavamos obter resultados melhores quando combinassemos ambos os conjuntos de dados em um só, mas
os resultados ficaram bem similares. Acreditamos que o resultado continua similar pois as propriedades
dadas pelas imagens da persistência possuem uma correlação com propriedades já conhecidas de proteínas, 
sendo assim a adição de novas propriedades não melhora o algoritmo. 

Os parâmetros para os algoritmos são os seguintes:
\begin{itemize}
    \item GBoost: n\_estimators $\in [100,300,500,700]$, min\_samples\_split $\in [2,5,10,15]$;
    \item Árv. Dec. I: n\_estimators $=689$, max\_features $=0.2$, max\_depth $=86$;
    \item Árv. Dec. II: n\_estimators $=500$, max\_depth $=100$, max\_features $=0.3$;
    \item Regressão Lin. c/ Reg: valor de alfa em $[0.005, 5000]$ (de $10$ em $10$) escolhido com 
     cross validation de 5 folds. 
\end{itemize}

\subsubsection{Segundo método}

Para o segundo método, utilizamos apenas as imagens de persistência para o treinamento. Obtivemos
um resultado similar ao dos modelos treinados com as propriedades das proteínas apenas. Os resultados
podem ser vistos nas tabelas a seguir. Note também como os resultados melhoram quando aumentamos
a variância utilizadada para as imagens de persistência. 

\begin{table}[htpb!]
\centering
\caption{Resultados dos modelos treinados utilizando apenas as imagens de persistência com variância
$0.1$}
\label{tab:pionly01}
\begin{tabular}{@{}ccccc@{}}
\toprule
Modelo                 & MSE    & RMSE   & Erro Percentual (\%) & $R^2$  \\ \midrule
Regressão Linear       & 0,2734 & 0,5229 & 14,19                & 0,3322 \\
GBoost                 & 0,2435 & 0,4935 & 13,39                & 0,4053 \\
Árvore de Decisão I    & 0,2490 & 0,4990 & 13,54                & 0,3917 \\
Árvore de Decisão II   & 0,2485 & 0,4985 & 13,53                & 0,3931 \\
Regressão Lin. c/ Reg. & 0,2723 & 0,5218 & 14,16                & 0,3350 \\
GBoost Ótimo           & 0,2450 & 0,4950 & 13,43                & 0,4017 \\ \bottomrule
\end{tabular}
\end{table}
\begin{table}[htpb!]
\centering
\caption{Resultados dos modelos treinados utilizando apenas as imagens de persistência com variância
$0.3$}
\label{tab:pionly03}
\begin{tabular}{@{}ccccc@{}}
\toprule
Modelo                 & MSE      & RMSE    & Erro Percentual (\%) & $R^2$      \\ \midrule
Regressão Linear       & 438,3102 & 20,9359 & 568,14               & -1069,5076 \\
GBoost                 & 0,2364   & 0,4863  & 13,20                & 0,4225     \\
Árvore de Decisão I    & 0,2417   & 0,4917  & 13,34                & 0,4096     \\
Árvore de Decisão II   & 0,2419   & 0,4919  & 13,35                & 0,4091     \\
Regressão Lin. c/ Reg. & 0,2649   & 0,5146  & 13,97                & 0,3531     \\
GBoost Ótimo           & 0,2305   & 0,4801  & 13,03                & 0,4371     \\
\bottomrule
\end{tabular}
\end{table}
\begin{table}[htpb!]
\centering
\caption{Resultados dos modelos treinados utilizando apenas as imagens de persistência com variância
$0.5$}
\label{tab:pionly05}
\begin{tabular}{@{}ccccc@{}}
\toprule
Modelo                 & MSE    & RMSE   & Erro Percentual (\%) & $R^2$  \\ \midrule
Regressão Linear       & 0,2639 & 0,5137 & 13,94 & 0,3555 \\
GBoost                 & 0,2327 & 0,4824 & 13,09 & 0,4316 \\
Árvore de Decisão I    & 0,2409 & 0,4908 & 13,32 & 0,4117 \\
Árvore de Decisão II   & 0,2403 & 0,4902 & 13,30 & 0,4130 \\
Regressão Lin. c/ Reg. & 0,2594 & 0,5093 & 13,82 & 0,3665 \\
GBoost Ótimo           & 0,2328 & 0,4825 & 13,09 & 0,4314 \\ \bottomrule
\end{tabular}
\end{table}
\begin{table}[htpb!]
\centering
\caption{Resultados dos modelos treinados utilizando apenas as imagens de persistência com variância
$0.7$}
\label{tab:pionly07}
\begin{tabular}{@{}ccccc@{}}
\toprule
Modelo                 & MSE    & RMSE   & Erro Percentual (\%) & $R^2$  \\ \midrule
Regressão Linear       & 0,2546 & 0,5046 & 13,69 & 0,3781 \\
GBoost                 & 0,2342 & 0,4839 & 13,13 & 0,4281 \\
Árvore de Decisão I    & 0,2379 & 0,4877 & 13,24 & 0,4190 \\
Árvore de Decisão II   & 0,2376 & 0,4874 & 13,23 & 0,4197 \\
Regressão Lin. c/ Reg. & 0,2535 & 0,5035 & 13,66 & 0,3809 \\
GBoost Ótimo           & 0,2276 & 0,4770 & 12,95 & 0,4442 \\ \bottomrule
\end{tabular}
\end{table}
Observamos que o melhor resultado é para o GBoost ótimo com variância $0,7$. Como o GBoost
é um método de árvores de decisão, ele nos dá a importância de cada uma das propriedades
dos vetores. Vamos fazer uma análise dessas propriedades. 

Primeiro, precisamos selecionar um número significativo de propriedades, já que algumas são
mais importantes que as outras. Cada propriedade tem um número associado a ela, dado pelo
algoritmo GBoost. A soma de todas esses valores da 1. Na Figura~\ref{fig:knee_plot} temos 
o logaritmo do valor da soma acumulada da propriedade com maior valor de importância até a última. 
\begin{figure}
    \centering
    \includegraphics[width=0.7\textwidth]{images/plt_ankle.pdf}
    \caption{Logaritmo da soma acumulada em relação à importância das propriedades dado pelo algoritmo
    GBoost.}
    \label{fig:knee_plot}
    \fautor
\end{figure}
Note como o plot dobra ao redor do número 50. Isso significa que as 50 primeiras propriedades são
as mais importantes, já contribuem mais para a soma acumulada. Pela construção das imagens
de persistência, temos que cada propriedade está relacionada com uma certa região da vetorização
do diagrama. Portanto, podemos localizar as regiões dos diagramas de persistência que 
aparecem nas propriedades. Dentre as 50 propriedades mais importantes, temos que a maioria
vem de pontos dos diagramas de dimensão 1 e correspondem a pontos com nascimentos logo
no início da filtração e com persistência baixa, como pode ser visto nas Figuras~\ref{fig:count_pd}
e~\ref{fig:count_pd2}
\begin{figure}[htpb!]
    \begin{minipage}{0.5\textwidth}
        \centering
        \includegraphics[width=1.1\textwidth]{images/heatmap_1.pdf}
        \caption{Heatmap das regiões dos diagramas de persistência de dimensão 1 que aparecem
        nas primeiras 50 propriedades. Eixo x representa nascimento, enquanto que o eixo y 
        representa a persistência.}
        \label{fig:count_pd}
        \fautor
    \end{minipage}
    \begin{minipage}{0.5\textwidth}
        \centering
        \includegraphics[width=1.1\textwidth]{images/heatmap_2.pdf}
        \caption{Heatmap das regiões dos diagramas de persistência de dimensão 2 que aparecem
        nas primeiras 50 propriedades. Eixo x representa nascimento, enquanto que o eixo y 
        representa a persistência.}
        \label{fig:count_pd2}
        \fautor
    \end{minipage}
\end{figure}
 
Dentre as propriedades mais importantes podemos analisar também a que conjunto de átomos elas estão 
associadas quando os diagramas de persistência foram calculados. Observamos na Figura~\ref{fig:plt_dim}
que os átomos associados aos ciclos que mais aparecem são os encontrados nos diagramas de persistência
calculados utilizando apenas carbono e nitrogênio. Os autores de~\cite{Cang2018} afirmam que 
ciclos associados a esses átomos representam propriedades hidrofóbicas e hidrofílicas, 
características que influenciam diretamente na estabilidade da proteína.  

\begin{figure}[htpb!]
    \centering
    \includegraphics[width=0.99\textwidth]{images/plt_dim.pdf}
    \caption{Número de ciclos associados para as top 50 propriedades e seus respectivos 
        diagramas de persistência.}
    \label{fig:plt_dim}
    \fautor
\end{figure}

\section{Analisando a energia total - Proteínas II}\label{sec:predrmsd}

Em \cite{Rubenstein2018} eles analisam a eficácia do \textit{Rosetta} e \textit{Amber}, dois softwares
para modelagem de macromoléculas. Dado uma proteína obtida do Protein Data Bank (PDB), por exemplo a proteína
de ID 1T2I, eles geraram novas moléculas usando amostragem ab-inition com viés e sem viés seguido por
uma amostragem paralela loophash. Após isso, essas amostras foram sujeitas à minimização no
backbone (átomos C-$\alpha$) e cadeias laterais (grupo-R) usando o protocolo talaris2014 e o minimzador
LBFGS. Então com os átomos C-$\alpha$ apenas, o RMSD foi calculado para todos as decoys (moléculas
geradas pelo software).

\begin{figure}[!htbp]
    \centering
    \includegraphics[width=0.7\textwidth]{images/relatorio/1t2i_tunnel.png}
    \caption{Panorama de energia para decoys modeladas em relação à proteína 1T2I.}
    \label{fig:1t2iland}
    \fautor
\end{figure}

Para cada decoy existe um score de energia associado, que é a função score minimizada pelo \textit{Rosetta}.
Com esse valor podemos plotar o panorama de energia para cada proteína, como na Figura~\ref{fig:1t2iland}.
O formato ideal seria o de um túnel, já que RMSD baixo corresponderia a uma energia normalizada baixa idealmente.

O score de energia dado por \textit{Rosetta} é normlizado usando a seguinte fórmula
\begin{equation}
    E_{i(norm)} = \frac{E_i - E_{\min}}{E_{95th} - E_{5th}},
\end{equation}
em que $E_{95th}$ é o $95$-ésimo percentil e $E_{5th}$ é o quinto percentil.

\subsection{Análise de falso mínimos}
Dados as moléculas geradas pelo software, podemos ranquear cada decoy de acordo com sua energia normalizada
e RMSD. Ranqueamos o conjunto de decoys da menor para a maior energia, por exemplo uma molécula com energia
de 0.3 está acima de outra com energia 0.5 no ranking. Na Tabela~\ref{tab:protrank} temos o top 5 decoys
das proteínas geradas a partir da 1T2I.

\begin{table}[!htbp]
    \centering
    \caption{Rank mostrando as top 5 decoys em relação a 1T2I.}
    \label{tab:protrank}
    \begin{tabular}{@{}ccc@{}}
        \toprule
        Rank & Normalized Energy & RMSD  \\
        \midrule
        1    & 0.000             & 2.233 \\
        2    & 0.023             & 1.37  \\
        3    & 0.025             & 2.395 \\
        4    & 0.057             & 2.004 \\
        5    & 0.061             & 2.356 \\
        \bottomrule
    \end{tabular}
\end{table}

Como mencionado anteriormente, \textit{Rosetta} tenta minimizar uma função score de energia de forma
que energia baixa corresponde a um valor baixo do RMSD. Dizemos que uma decoy é um falso mínimo
se está no top 10 moléculas do ranking de acordo com a definição acima e também possui um RMSD maior
do que $5$.

\subsubsection{VAE e ciclos ótimos}

Para analisar os falsos mínimos, utilizamos homologia persistente \cite{Edelsbrunner2002} para extrair
informações biológicas, como hidrofobicidade, juntamente com outras ferramentas topológicas \cite{Cang2017}.

Para cada ponto no diagrama de persistência existe um ciclo, um representante para sua respectiva classe
de homologia, que possui propriedades geométricas dos dados. Apesar disso, os ciclos não possuem o
verdadeiro tamanho do correspondente buraco $n$-dimensional, com respeito ao número de arestas. Portanto,
o problema de encontrar o ciclo ótimo em relação ao número de arestas é muito interessante, já que assim
podemos representar as propriedades topológicas de maneira muito mais fiel \cite{Escolar2015}.

Por um lado ciclos ótimos codificam muita informação, por outro lado é muitas vezes difícil analisa-los
de forma coesa. Portanto, nós propomos um método similar a \cite{Obayashi2018}. Primeiro vetorizamos
os diagramas de persistência utilizando imagens de persistência \cite{Adams2017} e após isso treinamos
um autoencoder variacional básico \cite{kingma2013} para extrair as regiões mais importantes
do diagrama de persistência. Então realizamos uma análise inversa, em que para
cada região existem pontos associados no diagrama de persistência e dessa forma seus
respectivos ciclos. Então, para cada conjunto de ciclos, somamos todos os átomos correspondentes de
cada ciclo, por xemplo, soma de todos os átomos de carbono de todos os ciclos. Nas próximas subseções
mostramos os resultados e parâmetros utilizados para gerar os diagramas de persistência, imagens
de persistência e hiperparâmetros para o treinamento do VAE.

\subsubsection{Resultados}

Selecionamos as proteínas de ID 2QY7 e 1T2I para análise. A primeira contém vários falsos mínimos,
enquanto o último não possuim nenhum falso mínimo.

Para a proteína 2QY7 calculamos os diagramas de persistêcnai para os top 100 decoys e plotamos a
som na Figura~\ref{fig:cyc2qy7}.

\begin{figure}[!htbp]
    \centering
    \includegraphics[width=0.7\textwidth]{images/relatorio/cyc2qy7.png}
    \caption{Soma dos átomos de carbono que compõem os ciclos do
            1º diagrama de persistência das decoys da 2QY7.}
    \label{fig:cyc2qy7}
    \fautor
\end{figure}

Para cada decoy foi calculado dois diagramas de persistência, um para a dimensão 1 e outro para a dimensão 2.
Em cada uma das dimensões usamos duas nuvens de pontos, a primeiro composta apenas pelos átomos C-$\alpha$
das moléculas e a outra composta apenas pelos átomos de nitrogênio e oxigênio.

Note que quando usamos apenas os átomos de carbono, a maioria dos falsos mínimos ficam agrupados em um intervalo
pequeno, como pode ser visto na Figura~\ref{fig:cyc2qy7}, e por outro lado com a outra nuvem de pontos os valores
ficaram espalhados, indicando que para esta proteína é melhor usar os átomos C-$\alpha$ para análise de falsos
mínimos.

Já para a proteína 1T2I um fenômeno similar acontece, como pode ser visto na Figura~\ref{fig:nocyc}. As top decoys
ficam em um intervalo menor, enquanto outras moléculas estão espalhadas pelo intervalo.
\begin{figure}[!htbp]
    \centering
    \includegraphics[width=0.99\textwidth]{images/relatorio/NOcyc.png}
    \caption{Sum of nitrogen (left) and oxygen (right) atoms composing the cycles of the 1st PD from 1T2I decoys.}
    \label{fig:nocyc}
    \fautor
\end{figure}

É importante notar que os ciclos da proteína com um panorâma de energia bom (1T2I) foram melhor caracterizados
pelos átomos de nitrogênio e oxigênio, enquanto que para a outra proteína, os átomos C-$\alpha$ caracterizaram melhor.

\subsubsection{Parâmetros}

Calcumos os primeiro e segundo diagramas de persistência usando a filtração alpha, onde o raio de cada átomo
era o raio de Van der Waals. Para os ciclos ótimos o software \textit{optiperslp} foi utilizado. As imagens
de persistência foram criadas utilizando a linguagem python e o pacote persim \cite{scikittda2019} com
os seguintes parâmetros: tamanho da imagem (pixel) = $(10,10)$, variância $=1$, e a função peso é a padrão
sugerida em \cite{Adams2017}.

Para o treinamento do VAE, 75 imagens de persistência foram utilizadas para o trienamento e 25 para o test. O
número de épocas é 300 e taxa de aprendizado igual a $0.0001$. O algoritmo de otimização utilizado foi o Adam.

O número de regiões das imagens de persistência selecionadas foi 5, ou seja, 5 regiões de 100 com
os maiores valores.

\subsection{Prevendo o RMSD}

Ao invés de usar a estrutura topologica dada pelos diagramas de persistência e os respectivos ciclos ótimos para
estudar os falsos mínimos, utilizamos as imagens de persistência de várias decoys de diversas proteínas diferentes
em algorimos de machine learning, como regressão linear, árvores de decisão, redes neurais e regressão linear
com regularização.

Escolhemos as proteínas 1T2I e 2NQW para testar e um outro conjunto de proteínas para o treinamento (proteínas
que contêm pelo menos um falso mínimo). O top 10 para ambas as proteínas pode ser visto na Figura~\ref{fig:truermsd}.

\begin{figure}[!htbp]
    \centering
    \includegraphics[width=0.5\textwidth]{images/relatorio/true_rmsd.png}
    \caption{Value of the RMSD for each decoy in the top 10. There are no false minima for the protein 1T2I,
            meanwhile there are 7 false minima for the protein 2NQW.}
    \label{fig:truermsd}
    \fautor
\end{figure}

\subsubsection{Resultados}

Para podermos comparar os resultados dos teste utilizando imagens de persistência, 
treinamos os mesmos algoritmos no conjunto de propriedades de proteínas dadas pelo \textit{Rosetta} 
quando desenvolvendo um novo decoy. As propriedades são dadas por
\begin{center}
    fa\_dun, fa\_elec, fa\_intra\_rep, hbond\_sc,

    fa\_rep, fa\_sol, hbond\_bb\_sc, hbond\_lr\_bb,

    hbond\_sr\_bb, omega, p\_aa\_pp, pro\_close, rama.
\end{center}
Quando treinamos os regressores com essas propriedades, obtemos as seguintes figuras. 
Na Figura~\ref{fig:res1t2i} são os resultados para a proteínas 1T2I e na Figura~\ref{fig:res2nqw} 
os resultados para 2NQW.

\begin{figure}[!htbp]
    \centering
    \includegraphics[width=.7\linewidth]{images/relatorio/res1t2i.png}
    \caption{Proteína 1T2I. RMSD previsto x RMSD verdadeiro para o top 10 dados os 
             regressores treinados em outras proteínas.}
    \label{fig:res1t2i}
    \fautor
\end{figure}

\begin{figure}[!htbp]
    \centering
    \includegraphics[width=.7\linewidth]{images/relatorio/res2nqw.png}
    \caption{Proteína 2NQW. RMSD previsto x RMSD verdadeiro para o top 10 dados os 
             regressores treinados em outras proteínas.}
    \label{fig:res2nqw}
    \fautor
\end{figure}

Agora pode analisar os modelos usando imagens de persistênia no treinamento. Na Tabela~\ref{run_numb} temos
os parâmetros utilizados para diversos testes. A coluna \textbf{Pixel} mostra o tamanho das imagens, $n$
signifca $(n,n)$. \textbf{\# Teste} é uma identificação para os resultados. Para cada teste três diagramas
de persistência foram calculados, somente os átomos C-$\alpha$, os átomos N e O e por fim todos os átomos
menos os de hidrogênio.

\begin{table}[!htbp]
    \centering
    \caption{Alguns dos testes feitos para obter as imagens de persistência e usa-las para o
             treinamento.}
    \label{tab:run_numb}
    \begin{tabular}{@{}cccc@{}}
    \toprule
    \textbf{\# Teste} & \textbf{Pixel} & \textbf{Variância} & \textbf{Dimensão PD} \\
    \midrule
    1                   & 10                  & 0.3             & 1                     \\
    2                   & 10                  & 0.5             & 1                     \\
    3                   & 10                  & 0.6             & 1                     \\
    4                   & 10                  & 0.8             & 1                     \\
    5                   & 10                  & 1.0             & 1                     \\
    6                   & 10                  & 1.2             & 1                     \\
    7                   & 3                   & 1.0             & 1                     \\
    8                   & 5                   & 1.0             & 1                     \\
    9                   & 50                  & 0.3             & 1                     \\
    10                  & 50                  & 1.0             & 1                     \\
    11                  & 100                 & 0.3             & 1                     \\
    12                  & 100                 & 1.0             & 1                     \\
    %18                  & 6                   & 0.3             & 1                     \\
    %19                  & 6                   & 0.5             & 1                     \\
    %20                  & 6                   & 0.6             & 1                     \\
    %21                  & 6                   & 0.8             & 1                     \\
    %22                  & 6                   & 1.0             & 1                     \\
    %23                  & 6                   & 1.2             & 1                     \\
    %24                  & 8                   & 0.3             & 1                     \\
    %25                  & 8                   & 0.5             & 1                     \\
    %26                  & 8                   & 0.6             & 1                     \\
    %27                  & 8                   & 0.8             & 1                     \\
    %28                  & 8                   & 1.0             & 1                     \\
    %29                  & 8                   & 1.2             & 1                     \\
    \bottomrule
\end{tabular}
\end{table}

Treinamos os mesmos regressores como anterioemnte para varias proteínas. Definimos então $4$ métricas diferentes
para selecionar o melhor regressor com respeito a cada uma. As métricas são:
\begin{itemize}
    \item $R^2$ score: medida estatística para medir o quão perto os dados estão da linha de regressão;
    \item MSE: Erro quadrático médio;
    \item RMSE: Raíz do erro quadrático médio;
    \item Acurácia binária: Converte cada RMSD para 0 ou 1 usando a seguinte regra: se o RMSD é maior que
          $5$ então 0, senão 1.
\end{itemize}

A Tabela~\ref{tab:bestruns} mostra os melhores experimentos para cada métrica usando imagens de persistência na hora do treinamento.
\begin{table}[!htbp]
    \centering
    \caption{Best parameters for each metric}
    \label{tab:bestruns}
    \begin{tabular}{@{}cccccc@{}}
    \toprule
    \textbf{Métrica} & \textbf{Regressor} & \textbf{Pixel} & \textbf{Variância} &
    \textbf{Lista de átomos}\footnote{Atomos utilizados para calcular os diagramas de persistência. "todo" significa
    que todos os átomos menos os de hidrogênio foram usados para os PD's.}
     & Score médio
    \\
    \midrule
    $R^2$           & Redes neurais     & 100       & 1.0             & C     &  $-5.780$ \\
    MSE             & Redes neurais     & 100       & 1.0             & C     &  $8.299$  \\
    RMSE            & Regressão com regu.   & 10        & 1.2             & todo &  $2.599$  \\
    Acurácia Binária & GBoost             & 10        & 0.6             & N,O   &  $0.657$  \\
    \bottomrule
    \end{tabular}
\end{table}
Por outro lado, a Tabela~\ref{tab:rosregr} mostra os melhores regressores treinados 
nas propriedades dadas pelo \textit{Rosetta}. 
\begin{table}[!htbp]
    \centering
    \caption{Melhores regressores treinados com as propriedades das proteínas}.
    \label{tab:rosregr}
    \begin{tabular}{@{}ccc@{}}
        \toprule
        \textbf{Métrica} & \textbf{Regressor} & \textbf{Score médio} \\ \midrule
        $R^2$           & Random Forest II   & -13.706        \\
        MSE             & Random Forest II   & 10.113         \\
        RMSE            & Random Forest II   & 2.707          \\
        Acurácia binária & Regressão com regu. & 0.586          \\ \bottomrule
    \end{tabular}
\end{table}

Os regressores treinados com imagens de persistência obtiveram melhores resultados 
do que os treinados utilizando as propriedades que o \textit{Rosetta} para todas as métricas.
Podemos ver na Figura~\ref{fig:1t2i_binary} e ~\ref{fig:2nqw_binary} que os modelos baseados em 
imagens de persistência possuem uma maior acurácia binária. O método baseado em propriedades topológicas
pode ser estendido. Devido a sua natureza e similaridade com imagens, pode-se utilizar redes neurais 
convolucionais para o treinamento. 
\begin{figure}[!htbp]
    \centering
    \includegraphics[width=0.99\textwidth]{images/relatorio/1t2i_binary.png}
    \caption{RMSD previsto x RMSD verdadeiro para o top 10 decoys da proteína 1T2I dados os 
             regressores com a melhor acurácia binária no conjunto de validação.}
    \label{fig:1t2i_binary}
    \fautor
\end{figure}

\begin{figure}[!htbp]
    \centering
    \includegraphics[width=0.99\textwidth]{images/relatorio/2nqw_binary.png}
    \caption{RMSD previsto x RMSD verdadeiro para o top 10 decoys da proteína 2NQW dados os 
             regressores com a melhor acurácia binária no conjunto de validação.}
    \label{fig:2nqw_binary}
    \fautor
\end{figure}

Este trabalho mostra que usar imagens de persistência é melhor para as tarefas de predição do RMSD
para proteínas não vistas anteriormente. Os algortimos treinados podem ser usados como uma função
para o \textit{Rosetta} utilizar na hora dos passos de minimização no desenvolvimento de novas proteínas.

O Jupyter Notebook \cite{Kluyver2016} está disponível online com a lista completa de proteínas (ID's) utilizadas 
no treinamento e teste, assim como com o código para a análise de resultados dos modelos. Os arquivos
podem ser baixados \href{https://drive.google.com/file/d/160DZgRiPwsHNaTzasQxd2VaIXCUkiLZG/view?usp=sharing}{aqui}
(\url{https://bit.ly/2XUjat2}).



\chapter{Conclusão}
\label{chapter:conclusao}
Este trabalho propôs a apresentação de homologia persistente, desde os princípios
básicas a teoria, assim como aplicações diretas que produziram resultados 
comparáveis ao estado da arte.

O problema de enovelamente de proteína é algo que precisa ser estudado e novos
métodos precisam ser discutidos. Nesta dissertação apresentamos novos métodos
para o estudo do problema e obtivemos resultados similares aos de estado
da arte propostos por grupos de renome internacional. O conteúdo
apresentado é fruto de um trabalho interdisciplinar e mostra também
o potencial da análise topológica de dados para tentar resolver outros
problemas de biologia. 

O aluno também desenvolveu diversos pacotes, contribuindo diretamente tanto
para a comunidade de topologia aplicada como para a de bioinformática.
A lista de pacotes desenvolvidos é a seguinte:
\begin{itemize}
    \item \href{https://github.com/chronchi/MapperMDS.jl}{MapperMDS.jl}: 
        uma implementação do mapper em Julia.
    \item \href{https://github.com/chronchi/PersistenceImage.jl}{PersistenceImage.jl}:
        implementação da imagem de persistência em Julia. 
    \item \href{https://github.com/chronchi/ProteinPersistent.jl}{ProteinPersistent.jl}:
        pacote que faz chamada do Bio.PDB e ripser do python para o cálculo dos diagramas
        de persistência de proteínas em Julia. 
    \item \href{https://github.com/chronchi/perscode}{perscode}: pacote de vetorização
        de diagramas de persistência descritos em~\cite{zielinski2018} na linguagem
        de programação python.
\end{itemize}

Todos os pacotes podem ser encontrados em \url{https://github.com/chronchi}. A dissertação,
códigos e arquivos tex podem ser acessados em \url{https://github.com/chronchi/dissertacao}.



% ---
% Finaliza a parte no bookmark do PDF, para que se inicie o bookmark na raiz
% ---
\bookmarksetup{startatroot}%
% ---

% ----------------------------------------------------------
% ELEMENTOS PÓS-TEXTUAIS
% ----------------------------------------------------------
\postextual

% ----------------------------------------------------------
% Referências bibliográficas
% ----------------------------------------------------------
\bibliography{references}

% ---------------------------------------------------------------------
% GLOSSÁRIO
% ---------------------------------------------------------------------

% Arquivo que contém as definições que vão aparecer no glossário
%\input{tex/glossario}
% Comando para incluir todas as definições do arquivo glossario.tex
%\glsaddall
% Impressão do glossário
%\printglossaries

% ----------------------------------------------------------
% Apêndices
% ----------------------------------------------------------

% ---
% Inicia os apêndices
% ---
\begin{apendicesenv}

   \chapter{Algoritmo \textit{standard} e funções auxiliares}
   \label{an:std_alg}
   \lstinputlisting{algoritmos/std_alg.jl}


\end{apendicesenv}
% ---


% ----------------------------------------------------------
% Anexos
% ----------------------------------------------------------

% ---
% Inicia os anexos
% ---
%\begin{anexosenv}

%\end{anexosenv}
% ---

\end{document}
