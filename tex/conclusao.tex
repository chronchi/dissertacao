Este trabalho propôs a apresentação de homologia persistente, desde os princípios
básicas a teoria, assim como aplicações diretas que produziram resultados 
comparáveis ao estado da arte.

O problema de enovelamente de proteína é algo que precisa ser estudado e novos
métodos precisam ser discutidos. Nesta dissertação apresentamos novos métodos
para o estudo do problema e obtivemos resultados similares aos de estado
da arte propostos por grupos de renome internacional. O conteúdo
apresentado é fruto de um trabalho interdisciplinar e mostra também
o potencial da análise topológica de dados para tentar resolver outros
problemas de biologia. 

O aluno também desenvolveu diversos pacotes, contribuindo diretamente tanto
para a comunidade de topologia aplicada como para a de bioinformática.
A lista de pacotes desenvolvidos é a seguinte:
\begin{itemize}
    \item \href{https://github.com/chronchi/MapperMDS.jl}{MapperMDS.jl}: 
        uma implementação do mapper em Julia.
    \item \href{https://github.com/chronchi/PersistenceImage.jl}{PersistenceImage.jl}:
        implementação da imagem de persistência em Julia. 
    \item \href{https://github.com/chronchi/ProteinPersistent.jl}{ProteinPersistent.jl}:
        pacote que faz chamada do Bio.PDB e ripser do python para o cálculo dos diagramas
        de persistência de proteínas em Julia. 
    \item \href{https://github.com/chronchi/perscode}{perscode}: pacote de vetorização
        de diagramas de persistência descritos em~\cite{zielinski2018} na linguagem
        de programação python.
\end{itemize}

Todos os pacotes podem ser encontrados em \url{https://github.com/chronchi}. A dissertação,
códigos e arquivos tex podem ser acessados em \url{https://github.com/chronchi/dissertacao}.
