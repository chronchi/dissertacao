A topologia é a área da matemática que estuda formas e geometrias 
dos objetos através de funções contínuas. Um exemplo clássico 
é o da rosquinha e a caneca. Esses dois objetos tão distintos na vida
real se tornam iguais do ponto de vista topológico, já que existem 
tranformações que os torcem e entortam até o outro,
mas não cortam e colam. Para um estudo mais sistemático e algébrico,
a topologia algébrica foi desenvolvida. Com ela associamos
grupos ou espaços vetoriais a espaços topológicos e assim podemos extrair
informações como componentes conexas, buracos e cavidades, além
de outros buracos n-dimensionais. 

O estudo topológico dos objetos matemáticos com a álgebra é estudado 
desde o início do século XX com Poincaré~\cite{Poincare1895}. Mas 
apenas recentemente começou a se desenvolver algoritmos, devido ao
advento dos dados e computadores para cálculos eficientes. 
Essa nova área, chamada de análise topológica de dados, possui 
várias ferramentas, como a homologia persistente~\cite{edelsbrunner2010computational}
e o mapper~\cite{mapper}. 

A homologia persistente estuda os invariantes topológicos de dados 
através da topologia algébrica. Para cada conjunto de dados,
podemos construir um complexo simplicial, objeto combinatório da 
matemática que codifica informações do conjunto geometricamente. Com esse complexo,
podemos construir uma filtração e então obter informações
dos buracos n-dimensionais do conjunto de dados através dos
grupos de homologia, que geralmente são espaços vetoriais, e que por fim
são codificados no diagrama de persistência. Essa técnica
pode ser considerada uma ferramenta de redução de dimensão, assim como
para a extração de propriedades geométricas
de dados em baixa dimensão, como no plano ou espaço 3D.

Devido a natureza da ferramenta, nesta dissertação mostramos algumas
aplicações em biologia. O problema de enovelamento de proteína
pode ser resumido com a seguinte expressão:
\begin{center}
    Sequência de aminoacidos $\rightarrow$ Estrutura $\rightarrow$ Função.
\end{center}
A questão fundamental é como encontrar uma estrutura estável com uma específica
função a partir de uma sequência de aminoácidos específica.\cite{Dill2008}
Vários métodos foram desenvolvidos envolvendo propriedades físicas e estatísticas
das proteínas para prever a estrutura de uma molécula dada uma sequência de aminoácidos.
O mais conhecido é o software Rosetta, desenvolvido por um grupo da Universidade 
de Washington em 1997.\cite{Simons1997}

Este software tem tido muito sucesso na predição de proteínas pequenas, mas não é
perfeito. Várias vezes ele apresenta proteínas não estáveis, o que em desenvolvimento
de proteínas em larga escala pode acarretar em custos mais altos quando testes são realizados
no laboratório. Em \cite{Rocklin2017} apenas algumas proteínas são escolhidas
para a fase de testes em laboratório após seu modelamento no Rosetta. O método
de escolha é dado por um algoritmo de machine learning, que prevê a estabilidade
das moléculas baseado em 110 propriedades por proteína.

Como aplicação nós desenvolvemos um novo método para a escolha
de proteínas para a fase de testes que não apenas prevê a estabilidade em nível
do estado da arte, mas também nos dá novas informações. Para cada proteína, temos
alguns diagramas de persistência. Cada ponto desse diagrama de persistência
está relacionado com alguma cadeia na classe de homologia que pode ser 
visualizada na molécula. Temos então essa nova informação geométrica
que pode auxiliar no design de novas proteínas. 

Uma outra aplicação é o desenvolvimento de uma função de energia para o software 
Rosetta baseado em outras proteínas. Essa nova função de energia foi treinada
para prever a distância da molécula simulada em relação a original. Diversas proteínas
do PDB (Protein Data Bank) foram usadas no treinamento.

A dissertação está dividida nos seguintes capítulos. No Capítulo~\ref{chapter:hp101}
temos uma introdução computacional de homologia persistente, apresentando as principais
filtrações e como obter as informações topológicas através do algoritmo de redução. 
No Capítulo~\ref{chapter:mph} temos a fundamentação teórica de homologia persistente,
através de ideias inspiradas na teoria de medida e categorias. O Capítulo~\ref{chapter:miscel}
contém algoritmos decorrentes da homologia persistente para a utilização no aprendizado
de máquinas e também há uma introdução do mapper. Já no Capítulo~\ref{chapter:aplicacoes}
apresentamos duas aplicações voltadas ao desenvolvimento de proteínas, mostrando resultados
novos e similares aos do estado da arte. E o Capítulo~\ref{chapter:conclusao} fecha a dissertação
com algumas conclusões.  

