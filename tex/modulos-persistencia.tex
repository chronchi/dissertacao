A homologia persistence teve seu ínicio em uma intersecção entre as ciências da computação 
e a matemática. Os primeiros artigos mostravam algoritmos sobre espaços topológicos simples,
como esferas \cite{Edelsbrunner2000}. No entanto, a teoria foi se desenvolvendo 
ao longo dos anos ao ponto em que as linguagens utilizadas para tratar da homologia persistente 
é a teoria de categorias conjuntamente com a teoria de representações \cite{Chazal2016}.

Neste capítulo tratamos do desenvolvimento da homologia persistente sob a luz dessas linguagens. 
Na primeira seção definimos o que são os módulos de persistência e suas relações com os diagramas de 
persistência. Na segunda seção descrevemos a medida retangular, usada para abstrair o conceito 
de diagrama de persistência e poder estudar o quão \textit{tame} ele o é. Apresentamos na terceira seção
alguns exemplos do comportamento dos módulos de persistência e exemplos. A quarta seção é fundamental,
pois mostramos como comparar dois módulos de persistência, através do \textit{interleaving}. E finalmente,
apresentamos o teoria de isometria e mostramos uma das implicações com a teoria desenvolvido neste capítulo. 

\section{Módulos de persistência e decomposições}
Nesta seção iremos definir os módulos de persistência, apresentar teoremas de decomposição dos módulos 
e introduzir a notação de quiver, que será utilizada para as próximas seções e demonstrações de 
outros resultados. 

Fixaremos aqui o corpo $\mathbf{k}$ para todos os espaços vetoriais apresentados neste texto. 

\begin{defi}
    Um módulo de persistência $\mathfrak{V}$ sobre os números reais $\mathbb{R}$ é uma família indexada 
    sobre $\mathbb{R}$ de espaços vetoriais 
    \begin{equation*}
        (V_t \mid t \in \mathbb{R}), 
    \end{equation*} 
    e uma família de aplicações lineares duplamente indexadas
    \begin{equation*}
        (v_t^s \colon V_s \to V_t \mid s \leq t) 
    \end{equation*}
    que satisfazem a seguinte relação de composição
    \begin{equation*}
        v_t^s \circ v_s^r = v_t^r,
    \end{equation*} 
    em que a função $v^r_r$ é considerada a função identidade. 
\end{defi}

O módulo de persistência pode ser visto como um funtor entre a categoria dos números reais com o morfismo
$s \to t$, em que $s \leq t$ e a categoria de espaços vetoriais. 

Vamos dar um exemplo de módulo de persistência que se encontra no contexto de análise topológica de dados. 
Seja $X$ um espaço vetorial e $f \colon X \to \mathbb{R}$ uma função, não necessariamente contínua e 
considere os conjuntos de nível
\begin{equation*}
    X^t = (X,f)^t = \Set{x \in X \mid f(x) \leq t}.
\end{equation*}

Temos uma sequência de conjuntos encaixados, $X^t$ com $t \in \mathbb{R}$, ou seja, existe uma função 
inclusão $\iota_t^s \colon X^s \hookrightarrow X^t$ que satisfaz trivialmente a lei de composição e 
existe uma função identidade. Chamamos esta sequência de conjuntos e funções de filtração de subníveis
de $(X,f)$, denotada por $\mathfrak{X}_{sub}$ ou $\mathfrak{X}^f_{sub}$.

Dada a sequência acima, podemos transforma-la em um módulo de persistência utilizando qualquer funtor
da categoria de espaços topológicos para a categoria de espaços vetoriais. Neste caso utilizamos 
o funtor de homologia $H = H_k(-, \mathbf{k})$ de dimensão $k$ com coeficientes em $\mathbf{k}$. Assim,
podemos definir o seguinte módulo de persistência $\mathfrak{V}$

\begin{equation*}
    V_t = H(X^t) \qquad v^s_t = H(\iota_t^s) \colon V_s \to V_t.
\end{equation*}
Podemos também escrever $\mathfrak{V} = H(\mathfrak{X}_{sub})$. 

Um exemplo na análise topológica de dados é quando $X$ é um complexo simmplicial finito e $X^t$ é um 
subcomplexo. Devido as propriedades dos complexos, existem finitos valores críticos onde há mudanças 
em $X$. Suponha que os valores sejam $a_1 < \dots < a_n$. Entao toda a informação do módulo de 
persistência é dada pela seguinte sequência de espaços vetoriais de dimensão finita

\begin{equation*}
    H(X^{a_1}) \to \dots \to H(X^{a_n}).
\end{equation*}

Neste caso, $H(\mathfrak{X}_{sub})$ admite uma descrição compacta, existe um algoritmo eficiente para 
o seu calculo e por último, a descrição é contínua com relação a $f$, ou seja, é estável sob uma 
métrica. 

A descrição mencionada acima é o diagrama de persistência ou barcode. A estrutura é dada por uma 
lista de intervalos da forma $[b,d) = [a_i, a_j)$ ou $[a_i, +\infty)$. Cada intervalo representa
um ciclo, uma propriedade, que nasce em $b$ e morre em $d$. 

Iremos mostrar aqui que é possível associar um diagram de persistência para módulos de 
persistência $\mathfrak{V}$ \textit{q-tame}. Um módulo de persistência é \textit{q-tame} 
se 
\begin{equation*}
    r_t^s = \text{rank}(v_t^s) < \infty \text{ para } s < t.
\end{equation*}
Intuitivamente falando, um módulo é \textit{q-tame} se para todo quadrante que pegamos com a origem
na diagonal, existem finitos pontos do diagram de persistência neste quadrante como pode ser visto
na \autoref{fig:quad_finito}.

\begin{figure}
    \centering
    \includegraphics[width=0.5\textwidth]{images/placeholder.png}
    \caption{Exemplo de um diagram de persistência de um módulo de 
            persistência \textit{q-tame} com um quadrante em destaque.}
    \label{fig:quad_finito}
\end{figure}

\subsection{Indíces e posets} 
No início desta seção definimos o módulo de persistência com o conjunto de indíces sendo os reais. No 
entanto, é possível definir utilizando quaisquer conjuntos parcialmente ordenados da mesma forma que 
com os reais. Seja $\mathbf{T}$ um poset, a coleção de espaços vetoriais e aplicações lineares que 
satisfazem as leis de composição e identidade é chamada de $\mathbf{T}$-módulo de persistência, ou 
módulo de persistência sobre $\mathbf{T}$. 

Além disso, podemos restringir o poset $\mathbf{T}$ para um subconjunto $\mathbf{S} \subset \mathbf{T}$
de forma a obter o $\mathbf{S}$-módulo de persistência, que são os espaços vetoriais e aplicações lineares
cujos indíces são elementos de $\mathbf{S}$. Esta é a restrição de $\mathfrak{V}$ em $\mathbf{S}$ e pode
ser denotada por $\mathfrak{V}_S$ ou $\left.\mathfrak{V}\right|_S$. 

\subsection{Categoria de módulos}

\subsection{Módulos Intervalares}

\subsection{Decomposição em módulos intervalares} 

\begin{propo}
    Seja $\mathfrak{I} = \mathbf{k}^J_T$ um módulo intervalor sobre $\mathbf{T} \subset \mathbb{R}$. 
    Então $\text{End}(\mathfrak{I}) = \mathbf{k}$. 
\end{propo}
\begin{proof}

\end{proof}

\begin{propo}
    Módulos intervalares são indecomponíveis. 
\end{propo}
\begin{proof}

\end{proof}

\begin{teo}{(Krull-Remak-Schmidt-Azumaya)}
    Suponha que um módulo de persistência $\mathbf{T} \subset \mathbb{R}$ pode ser escrito como soma 
    direta de módulos intervalores de duas formas diferentes
    \begin{equation*}
        \mathfrak{V} \simeq \bigoplus_{l \in L} \mathbf{k}^{J_l} \simeq \bigoplus_{m \in M} \mathbf{k}^{K_m},
    \end{equation*}
    então existe uma bijeção $\sigma \colon L \to M$ tal que $J_l = K_{\sigma(l)}$ para todo $l \in L$. 
\end{teo}
\begin{proof}

\end{proof}

\begin{teo}{(Gabriel, Auslander, Ringel-Tachikawa, Webb, Crawley-Boevey)}
    Seja $\mathfrak{V}$ um módulo de persistência sobre $\mathbf{T} \subset \mathbb{R}$. Então $\mathfrak{V}$
    pode ser decomposto como um soma direta de módulos intervalares sob as seguintes condições:
    \begin{itemize} 
        \item $\mathbf{T}$ é um conjunto finito;
        \item cada $V_t$ é um espaço vetorial de dimensão finita. 
    \end{itemize}
    Por outro lado, existe um módulo de persistência sob $\mathbb{Z}$ que não admite uma decomposição intervalar. 
\end{teo}  
\begin{proof}
    Detalhes podem ser vistos em \cite{Chazal2016}, página 22, \textbf{Teorema} 2.8.
\end{proof}

\subsection{Diagrama de persistência e decomposição}

\section{Medidas retangulares}

\section{Comportamento de módulos e exemplos}

\section{\textit{Interleaving}} 

\section{O teorema de isometria}
