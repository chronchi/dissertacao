O problema de enovalemento da proteína é a questão fundamental de como a sua sequência de aminoácidos
no plano se transforma em uma estrutura átomica tridimensional. Esta questão é essencial, pois
um melhor entendimento dessa situação pode levar ao desenvolvimento de novos remédios e também
uma melhora no combate de doenças. No entanto, continua sendo um grande desafio obter
uma estrutura estável da proteína a partir da sequência de aminoácidos.
Recentemente, alguns estudos \cite{Rocklin2017} desenvolveram novas proteínas usando o softaware
\textit{Rosetta}, que modela estruturas macromolecures. Existem alguns problemas no
desenvolvimento usando tal software, como proteínas modeladas que não são tão
estáveis sobre o processo de proteólise, que é a quebra da proteína em pedaços
menores. Pode-se contornar este problema através do uso de outras ferramentas avançadas, como
as encontradas em aprendizado de máquinas e análise topológica de dados.

Neste capítulo estudamos a estabilidade de proteínas sobre um score proposto
em \cite{Rocklin2017} utilizando imagens de persistência \cite{Adams2017}, descritas
no Capítulo \ref{chapter:miscel}, e vários algoritmos de aprendizado de máquinas
implementados em \cite{scikit-learn}. Mais detalhes são dados na Seção \ref{sec:stabprot}


Por outro lado podemos estudar a perfomance de algoritmos para modelagem computacional e
análise estrutural de proteínas, como \textit{Rosetta} e \textit{Amber}.
Em \cite{Rubenstein2018}, ambos os softwares sõa comparados em relação a proteína-energia.
Para uma proteína específica, eles geraram milhares de moléculas similares e calcularem
a raíz do erro quadrático médio em relação a proteína original.
Após isso, eles analisaram as móleculas de falso mínimo. Dada uma lista de proteínas
simuladas, elas são ranqueadas de acordo com suas energias normalizadas. Uma molécula
simulada é uma falsa mínima se está no top $10$ das moléculas no ranking a seu
RMSD é maior do que $5$. Eles observaram que o software \textit{Rosetta} gerou mais
proteínas de falso mínimo do que o \textit{Amber}. Na Seção \ref{sec:predrmsd}
analisamos a estrutura das proteínas dadas pelo \textit{Rosetta} utilizando ciclos ótimos
\cite{Escolar2015}, imagens de persistência \cite{Adams2017}, VAE's \cite{kingma2013} e
diversos algoritmos de machine learning utilizando o sklearn \cite{scikit-learn}. Nós
tentamos prever e apresentar uma nova função para ajudar o \textit{Rosetta} no seu
passo de otimização quando simulando novas moléculas.

\section{Estudando a estabilidade - Proteínas I}\label{sec:stabprot}



\section{Analisando a energia total - Proteínas II}\label{sec:predrmsd}

Em \cite{Rubenstein2018} eles analisam a eficácia do \textit{Rosetta} e \textit{Amber}, dois softwares
para modelagem de macromoléculas. Dado uma proteína obtida do Protein Data Bank (PDB), por exemplo a proteína
de ID 1T2I, eles geraram novas moléculas usando amostragem ab-inition com viés e sem viés seguido por
uma amostragem paralela loophash. Após isso, essas amostras foram sujeitas à minimização no
backbone (átomos C-$\alpha$) e cadeias laterais (grupo-R) usando o protocolo talaris2014 e o minimzador
LBFGS. Então com os átomos C-$\alpha$ apenas, o RMSD foi calculado para todos as decoys (moléculas
geradas pelo software).

\begin{figure}[!htbp]
    \centering
    \includegraphics[width=0.7\textwidth]{images/relatorio/1t2i_tunnel.png}
    \caption{Panorama de energia para decoys modeladas em relação à proteína 1T2I.}
    \label{fig:1t2iland}
\end{figure}

Para cada decoy existe um score de energia associado, que é a função score minimizada pelo \textit{Rosetta}.
Com esse valor podemos plotar o panorama de energia para cada proteína, como na Figura~\ref{fig:1t2iland}.
O formato ideal seria o de um túnel, já que RMSD baixo corresponderia a uma energia normalizada baixa idealmente.

O score de energia dado por \textit{Rosetta} é normlizado usando a seguinte fórmula
\begin{equation}
    E_{i(norm)} = \frac{E_i - E_{\min}}{E_{95th} - E_{5th}},
\end{equation}
em que $E_{95th}$ é o $95$-ésimo percentil e $E_{5th}$ é o quinto percentil.

\subsection{Análise de falso mínimos}
Dados as moléculas geradas pelo software, podemos ranquear cada decoy de acordo com sua energia normalizada
e RMSD. Ranqueamos o conjunto de decoys da menor para a maior energia, por exemplo uma molécula com energia
de 0.3 está acima de outra com energia 0.5 no ranking. Na Tabela~\ref{tab:protrank} temos o top 5 decoys
das proteínas geradas a partir da 1T2I.

\begin{table}[!htbp]
    \centering
    \caption{Rank mostrando as top 5 decoys em relação a 1T2I.}
    \label{tab:protrank}
    \begin{tabular}{@{}ccc@{}}
        \toprule
        Rank & Normalized Energy & RMSD  \\
        \midrule
        1    & 0.000             & 2.233 \\
        2    & 0.023             & 1.37  \\
        3    & 0.025             & 2.395 \\
        4    & 0.057             & 2.004 \\
        5    & 0.061             & 2.356 \\
        \bottomrule
    \end{tabular}
\end{table}

Como mencionado anteriormente, \textit{Rosetta} tenta minimizar uma função score de energia de forma
que energia baixa corresponde a um valor baixo do RMSD. Dizemos que uma decoy é um falso mínimo
se está no top 10 moléculas do ranking de acordo com a definição acima e também possui um RMSD maior
do que $5$.

\subsubsection{VAE e ciclos ótimos}

Para analisar os falsos mínimos, utilizamos homologia persistente \cite{Edelsbrunner2002} para extrair
informações biológicas, como hidrofobicidade, juntamente com outras ferramentas topológicas \cite{Cang2017}.

Para cada ponto no diagrama de persistência existe um ciclo, um representante para sua respectiva classe
de homologia, que possui propriedades geométricas dos dados. Apesar disso, os ciclos não possuem o
verdadeiro tamanho do correspondente buraco $n$-dimensional, com respeito ao número de arestas. Portanto,
o problema de encontrar o ciclo ótimo em relação ao número de arestas é muito interessante, já que assim
podemos representar as propriedades topológicas de maneira muito mais fiel \cite{Escolar2015}.

Por um lado ciclos ótimos codificam muita informação, por outro lado é muitas vezes difícil analisa-los
de forma coesa. Portanto, nós propomos um método similar a \cite{Obayashi2018}. Primeiro vetorizamos
os diagramas de persistência utilizando imagens de persistência \cite{Adams2017} e após isso treinamos
um autoencoder variacional básico \cite{kingma2013} para extrair as regiões mais importantes
do diagrama de persistência. Então realizamos uma análise inversa, em que para
cada região existem pontos associados no diagrama de persistência e dessa forma seus
respectivos ciclos. Então, para cada conjunto de ciclos, somamos todos os átomos correspondentes de
cada ciclo, por xemplo, soma de todos os átomos de carbono de todos os ciclos. Nas próximas subseções
mostramos os resultados e parâmetros utilizados para gerar os diagramas de persistência, imagens
de persistência e hiperparâmetros para o treinamento do VAE.

\subsubsection{Resultados}

Selecionamos as proteínas de ID 2QY7 e 1T2I para análise. A primeira contém vários falsos mínimos,
enquanto o último não possuim nenhum falso mínimo.

Para a proteína 2QY7 calculamos os diagramas de persistêcnai para os top 100 decoys e plotamos a
som na Figura~\ref{fig:cyc2qy7}.

\begin{figure}[!htbp]
    \centering
    \includegraphics[width=0.7\textwidth]{images/relatorio/cyc2qy7.png}
    \caption{Soma dos átomos de carbono que compõem os ciclos do
            1º diagrama de persistência das decoys da 2QY7.}
    \label{fig:cyc2qy7}
\end{figure}

Para cada decoy foi calculado dois diagramas de persistência, um para a dimensão 1 e outro para a dimensão 2.
Em cada uma das dimensões usamos duas nuvens de pontos, a primeiro composta apenas pelos átomos C-$\alpha$
das moléculas e a outra composta apenas pelos átomos de nitrogênio e oxigênio.

Note que quando usamos apenas os átomos de carbono, a maioria dos falsos mínimos ficam agrupados em um intervalo
pequeno, como pode ser visto na Figura~\ref{fig:cyc2qy7}, e por outro lado com a outra nuvem de pontos os valores
ficaram espalhados, indicando que para esta proteína é melhor usar os átomos C-$\alpha$ para análise de falsos
mínimos.

Já para a proteína 1T2I um fenômeno similar acontece, como pode ser visto na Figura~\ref{fig:nocyc}. As top decoys
ficam em um intervalo menor, enquanto outras moléculas estão espalhadas pelo intervalo.
\begin{figure}[!htbp]
    \centering
    \includegraphics[width=0.99\textwidth]{images/relatorio/NOcyc.png}
    \caption{Sum of nitrogen (left) and oxygen (right) atoms composing the cycles of the 1st PD from 1T2I decoys.}
    \label{fig:nocyc}
\end{figure}

É importante notar que os ciclos da proteína com um panorâma de energia bom (1T2I) foram melhor caracterizados
pelos átomos de nitrogênio e oxigênio, enquanto que para a outra proteína, os átomos C-$\alpha$ caracterizaram melhor.

\subsubsection{Parâmetros}

Calcumos os primeiro e segundo diagramas de persistência usando a filtração alpha, onde o raio de cada átomo
era o raio de Van der Waals. Para os ciclos ótimos o software \textit{optiperslp} foi utilizado. As imagens
de persistência foram criadas utilizando a linguagem python e o pacote persim \cite{scikittda2019} com
os seguintes parâmetros: tamanho da imagem (pixel) = $(10,10)$, variância $=1$, e a função peso é a padrão
sugerida em \cite{Adams2017}.

Para o treinamento do VAE, 75 imagens de persistência foram utilizadas para o trienamento e 25 para o test. O
número de épocas é 300 e taxa de aprendizado igual a $0.0001$. O algoritmo de otimização utilizado foi o Adam.

O número de regiões das imagens de persistência selecionadas foi 5, ou seja, 5 regiões de 100 com
os maiores valores.

\subsection{Prevendo o RMSD}

Ao invés de usar a estrutura topologica dada pelos diagramas de persistência e os respectivos ciclos ótimos para
estudar os falsos mínimos, utilizamos as imagens de persistência de várias decoys de diversas proteínas diferentes
em algorimos de machine learning, como regressão linear, árvores de decisão, redes neurais e regressão linear
com regularização.

Escolhemos as proteínas 1T2I e 2NQW para testar e um outro conjunto de proteínas para o treinamento (proteínas
que contêm pelo menos um falso mínimo). O top 10 para ambas as proteínas pode ser visto na Figura~\ref{fig:truermsd}.

\begin{figure}[!htbp]
    \centering
    \includegraphics[width=0.5\textwidth]{images/relatorio/true_rmsd.png}
    \caption{Value of the RMSD for each decoy in the top 10. There are no false minima for the protein 1T2I,
            meanwhile there are 7 false minima for the protein 2NQW.}
    \label{fig:truermsd}
\end{figure}

\subsubsection{Resultados}

Para podermos comparar os resultados dos teste utilizando imagens de persistência, 
treinamos os mesmos algoritmos no conjunto de propriedades de proteínas dadas pelo \textit{Rosetta} 
quando desenvolvendo um novo decoy. As propriedades são dadas por
\begin{center}
    fa\_dun, fa\_elec, fa\_intra\_rep, hbond\_sc,

    fa\_rep, fa\_sol, hbond\_bb\_sc, hbond\_lr\_bb,

    hbond\_sr\_bb, omega, p\_aa\_pp, pro\_close, rama.
\end{center}
Quando treinamos os regressores com essas propriedades, obtemos as seguintes figuras. 
Na Figura~\ref{fig:res1t2i} são os resultados para a proteínas 1T2I e na Figura~\ref{fig:res2nqw} 
os resultados para 2NQW.

\begin{figure}[!htbp]
    \centering
    \includegraphics[width=.7\linewidth]{images/relatorio/res1t2i.png}
    \caption{Proteína 1T2I. RMSD previsto x RMSD verdadeiro para o top 10 dados os 
             regressores treinados em outras proteínas.}
    \label{fig:res1t2i}
\end{figure}

\begin{figure}[!htbp]
    \centering
    \includegraphics[width=.7\linewidth]{images/relatorio/res2nqw.png}
    \caption{Proteína 2NQW. RMSD previsto x RMSD verdadeiro para o top 10 dados os 
             regressores treinados em outras proteínas.}
    \label{fig:res2nqw}
\end{figure}

Agora pode analisar os modelos usando imagens de persistênia no treinamento. Na Tabela~\ref{run_numb} temos
os parâmetros utilizados para diversos testes. A coluna \textbf{Pixel} mostra o tamanho das imagens, $n$
signifca $(n,n)$. \textbf{\# Teste} é uma identificação para os resultados. Para cada teste três diagramas
de persistência foram calculados, somente os átomos C-$\alpha$, os átomos N e O e por fim todos os átomos
menos os de hidrogênio.

\begin{table}[!htbp]
    \centering
    \caption{Alguns dos testes feitos para obter as imagens de persistência e usa-las para o
             treinamento.}
    \label{tab:run_numb}
    \begin{tabular}{@{}cccc@{}}
    \toprule
    \textbf{\# Teste} & \textbf{Pixel} & \textbf{Variância} & \textbf{Dimensão PD} \\
    \midrule
    1                   & 10                  & 0.3             & 1                     \\
    2                   & 10                  & 0.5             & 1                     \\
    3                   & 10                  & 0.6             & 1                     \\
    4                   & 10                  & 0.8             & 1                     \\
    5                   & 10                  & 1.0             & 1                     \\
    6                   & 10                  & 1.2             & 1                     \\
    7                   & 3                   & 1.0             & 1                     \\
    8                   & 5                   & 1.0             & 1                     \\
    9                   & 50                  & 0.3             & 1                     \\
    10                  & 50                  & 1.0             & 1                     \\
    11                  & 100                 & 0.3             & 1                     \\
    12                  & 100                 & 1.0             & 1                     \\
    %18                  & 6                   & 0.3             & 1                     \\
    %19                  & 6                   & 0.5             & 1                     \\
    %20                  & 6                   & 0.6             & 1                     \\
    %21                  & 6                   & 0.8             & 1                     \\
    %22                  & 6                   & 1.0             & 1                     \\
    %23                  & 6                   & 1.2             & 1                     \\
    %24                  & 8                   & 0.3             & 1                     \\
    %25                  & 8                   & 0.5             & 1                     \\
    %26                  & 8                   & 0.6             & 1                     \\
    %27                  & 8                   & 0.8             & 1                     \\
    %28                  & 8                   & 1.0             & 1                     \\
    %29                  & 8                   & 1.2             & 1                     \\
    \bottomrule
\end{tabular}
\end{table}

Treinamos os mesmos regressores como anterioemnte para varias proteínas. Definimos então $4$ métricas diferentes
para selecionar o melhor regressor com respeito a cada uma. As métricas são:
\begin{itemize}
    \item $R^2$ score: medida estatística para medir o quão perto os dados estão da linha de regressão;
    \item MSE: Erro quadrático médio;
    \item RMSE: Raíz do erro quadrático médio;
    \item Acurácia binária: Converte cada RMSD para 0 ou 1 usando a seguinte regra: se o RMSD é maior que
          $5$ então 0, senão 1.
\end{itemize}

A Tabela~\ref{tab:bestruns} mostra os melhores experimentos para cada métrica usando imagens de persistência na hora do treinamento.
\begin{table}[!htbp]
    \centering
    \caption{Best parameters for each metric}
    \label{tab:bestruns}
    \begin{tabular}{@{}cccccc@{}}
    \toprule
    \textbf{Métrica} & \textbf{Regressor} & \textbf{Pixel} & \textbf{Variância} &
    \textbf{Lista de átomos}\footnote{Atomos utilizados para calcular os diagramas de persistência. "todo" significa
    que todos os átomos menos os de hidrogênio foram usados para os PD's.}
     & Score médio
    \\
    \midrule
    $R^2$           & Redes neurais     & 100       & 1.0             & C     &  $-5.780$ \\
    MSE             & Redes neurais     & 100       & 1.0             & C     &  $8.299$  \\
    RMSE            & Regressão com regu.   & 10        & 1.2             & todo &  $2.599$  \\
    Acurácia Binária & GBoost             & 10        & 0.6             & N,O   &  $0.657$  \\
    \bottomrule
    \end{tabular}
\end{table}
Por outro lado, a Tabela~\ref{tab:rosregr} mostra os melhores regressores treinados 
nas propriedades dadas pelo \textit{Rosetta}. 
\begin{table}[!htbp]
    \centering
    \caption{Melhores regressores treinados com as propriedades das proteínas}.
    \label{tab:rosregr}
    \begin{tabular}{@{}ccc@{}}
        \toprule
        \textbf{Métrica} & \textbf{Regressor} & \textbf{Score médio} \\ \midrule
        $R^2$           & Random Forest II   & -13.706        \\
        MSE             & Random Forest II   & 10.113         \\
        RMSE            & Random Forest II   & 2.707          \\
        Acurácia binária & Regressão com regu. & 0.586          \\ \bottomrule
    \end{tabular}
\end{table}

Os regressores treinados com imagens de persistência obtiveram melhores resultados 
do que os treinados utilizando as propriedades que o \textit{Rosetta} para todas as métricas.
Podemos ver na Figura~\ref{fig:1t2i_binary} e ~\ref{fig:2nqw_binary} que os modelos baseados em 
imagens de persistência possuem uma maior acurácia binária. O método baseado em propriedades topológicas
pode ser estendido. Devido a sua natureza e similaridade com imagens, pode-se utilizar redes neurais 
convolucionais para o treinamento. 
\begin{figure}[!htbp]
    \centering
    \includegraphics[width=0.99\textwidth]{images/relatorio/1t2i_binary.png}
    \caption{RMSD previsto x RMSD verdadeiro para o top 10 decoys da proteína 1T2I dados os 
             regressores com a melhor acurácia binária no conjunto de validação.}
    \label{fig:1t2i_binary}
\end{figure}

\begin{figure}[!htbp]
    \centering
    \includegraphics[width=0.99\textwidth]{images/relatorio/2nqw_binary.png}
    \caption{RMSD previsto x RMSD verdadeiro para o top 10 decoys da proteína 2NQW dados os 
             regressores com a melhor acurácia binária no conjunto de validação.}
    \label{fig:2nqw_binary}
\end{figure}

Este trabalho mostra que usar imagens de persistência é melhor para as tarefas de predição do RMSD
para proteínas não vistas anteriormente. Os algortimos treinados podem ser usados como uma função
para o \textit{Rosetta} utilizar na hora dos passos de minimização no desenvolvimento de novas proteínas.

O Jupyter Notebook \cite{Kluyver2016} está disponível online com a lista completa de proteínas (ID's) utilizadas 
no treinamento e teste, assim como com o código para a análise de resultados dos modelos. Os arquivos
podem ser baixados \href{https://drive.google.com/file/d/160DZgRiPwsHNaTzasQxd2VaIXCUkiLZG/view?usp=sharing}{aqui}
(\url{https://bit.ly/2XUjat2}).

