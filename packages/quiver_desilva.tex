\documentclass{amsart}

%------------------------------------------------------------------------------------------
% quiver calculus commands
%------------------------------------------------------------------------------------------
%
% This is what we put in our preamble:
%

\newcommand{\qlen}{1em}
\newcommand{\Qlen}{1.5em}
\newcommand{\QQlen}{3em}

\newcommand{\qem}{\makebox[\qlen]{---}}

% short
\newcommand{\qno}{\makebox[\qlen]{---}}  
\newcommand{\qon}[1]{\makebox[\qlen]{$_{\phantom{}}\bullet_{#1}$}}
\newcommand{\qoff}[1]{\makebox[\qlen]{$_{\phantom{}}\circ_{#1}$}}

% medium
\newcommand{\Qno}{\makebox[\Qlen]{--{}--{}--}}
\newcommand{\Qon}[1]{\makebox[\Qlen]{$_{\phantom{}}\bullet_{#1}$}}
\newcommand{\Qoff}[1]{\makebox[\Qlen]{$_{\phantom{}}\circ_{#1}$}}

% long
\newcommand{\QQno}{\makebox[\QQlen]{--{}--{}--{}--{}--{}--}}
\newcommand{\QQon}[1]{\makebox[\QQlen]{$_{\phantom{}}\bullet_{#1}$}}
\newcommand{\QQoff}[1]{\makebox[\QQlen]{$_{\phantom{}}\circ_{#1}$}}


%-----------------------------------------------------------------
% \qem : draw a line.
%
% \qon{} : filled circle with subscript
% \qoff{} : empty circle with subscript
% \qno : no circle, but takes up the same space
%
% \Qno, \Qon, \Qoff : same but longer
% \QQno, \QQon, \QQoff : same but even longer
%



%------------------------------------------------------------------------------------------
\begin{document}
%------------------------------------------------------------------------------------------

\section*{Examples}

\bigskip
\noindent
Three different sizes for the labelled nodes.

short:
\[
\qoff{a} \qem \qon{b} \qem \qon{c} \qem \qoff{d}
\]

medium:
\[
\Qoff{a} \qem \Qon{b} \qem \Qon{c} \qem \Qoff{d}
\]

long:
\[
\QQoff{a} \qem \QQon{b} \qem \QQon{c} \qem \QQoff{d}
\]

\bigskip
\noindent
This is why we might use medium instead of short.

\medskip
short:
\[
\qoff{aa} \qem \qon{bb} \qem \qon{cc} \qem \qoff{dd}
\]

medium:
\[
\Qoff{aa} \qem \Qon{bb} \qem \Qon{cc} \qem \Qoff{dd}
\]

\bigskip
\noindent
The `missing circle' is used when aligning quiver diagrams.
%
\begin{align*}
&
\qoff{a} \qem \qon{b} \qem \qno \qem \qon{c} \qem \qoff{d}
\\
&
\qoff{a} \qem \qon{b} \qem \qon{x} \qem \qon{c} \qem \qoff{d}
\end{align*}

\bigskip
\bigskip
\noindent
Please see the source file to see how these examples were made.

%------------------------------------------------------------------------------------------
\end{document}
%------------------------------------------------------------------------------------------
